\documentclass[10pt,landscape]{article}
\usepackage{amssymb,amsmath,amsthm,amsfonts,bm}
\usepackage{multicol,multirow}
\usepackage{calc}
\usepackage{ifthen}
\usepackage[landscape]{geometry}
\usepackage[colorlinks=true,citecolor=blue,linkcolor=blue]{hyperref}
\usepackage{graphicx}
\graphicspath{ {./images/} }
\ifthenelse{\lengthtest { \paperwidth = 11in}}
    { \geometry{top=.2in,left=.2in,right=.2in,bottom=.2in} }
	{\ifthenelse{ \lengthtest{ \paperwidth = 297mm}}
		{\geometry{top=1cm,left=1cm,right=1cm,bottom=1cm} }
		{\geometry{top=1cm,left=1cm,right=1cm,bottom=1cm} }
	}
\pagestyle{empty}
\makeatletter
\renewcommand{\section}{\@startsection{section}{1}{0mm}%
                                {-1ex plus -.5ex minus -.2ex}%
                                {0.5ex plus .2ex}%x
                                {\normalfont\large\bfseries}}
\renewcommand{\subsection}{\@startsection{subsection}{2}{0mm}%
                                {-1explus -.5ex minus -.2ex}%
                                {0.5ex plus .2ex}%
                                {\normalfont\normalsize\bfseries}}
\renewcommand{\subsubsection}{\@startsection{subsubsection}{3}{0mm}%
                                {-1ex plus -.5ex minus -.2ex}%
                                {1ex plus .2ex}%
                                {\normalfont\small\bfseries}}
\makeatother
\setcounter{secnumdepth}{0}
\setlength{\parindent}{0pt}
\setlength{\parskip}{0pt plus 0.5ex}

\newcommand{\matr}[1]{\bm{#1}}
\newcommand{\vect}[1]{\bm{#1}}
\newcommand{\adj}{\operatorname{\textbf{adj}}}
\newcommand{\lspan}{\operatorname{span}}
\newcommand{\rank}{\operatorname{rank}}
\newcommand{\nullity}{\operatorname{nullity}}
\newcommand{\Ker}{\operatorname{Ker}}
\newcommand{\norm}[1]{\left\lVert#1\right\rVert}

\theoremstyle{definition}
\newcommand{\thistheoremname}{}
\newtheorem*{genericthm*}{\thistheoremname}
\newenvironment{namedthm*}[1]
{\renewcommand{\thistheoremname}{#1}\begin{genericthm*}}
{\end{genericthm*}}

% -----------------------------------------------------------------------

\title{MA2108 Cheatsheet 19/20 Sem 1 Midterm}

\begin{document}

\begin{center}
{\large MA2108 Cheatsheet 19/20 Sem 1 Midterm}\\{by Timothy Leong (format taken from Ning Yuan)}
\end{center}

\raggedright
\footnotesize

\begin{multicols}{3}

\setlength{\premulticols}{1pt}
\setlength{\postmulticols}{1pt}
\setlength{\multicolsep}{1pt}
\setlength{\columnsep}{2pt}

\section{The Real Numbers}

\begin{namedthm*}{Result 1}
    $2^{n-1} \leq n!~ \text{and} ~ n < 2^{n}~\forall n \in \mathbb{N}$
\end{namedthm*}

\begin{namedthm*}{Result 2}
    $n^{2} \leq 2^{n}~\forall n \in \mathbb{N} \backslash \{2,3,4\}$
\end{namedthm*}

\begin{namedthm*}{Well-ordering principle}
  Every nonempty subset $A$ of $\mathbb{N}$ has $a$ least (first) element,
i.e. there exists $p \in A$ such that $p \leq a$ for all $a \in A$
\end{namedthm*}

\begin{namedthm*}{Lemma 1.5.1}
~
    \begin{enumerate}
        \item If $c > 1$, then $c^{n} > c$ for every natural number $n \geq 2$.
        \item If $0 < c < 1$, then $c^{n} < c$ for every natural number $n \geq 2$.
    \end{enumerate}
\end{namedthm*}

\begin{namedthm*}{Theorem 1.5.2}
    For any non-zero number $a$, $a^{2} > 0$.
\end{namedthm*}
\begin{namedthm*}{Theorem 1.5.3}
    If $a \in \mathbb{R}$ is such that $0 \leq a< \varepsilon$ for every positive number $\varepsilon,$ then $a=0$
\end{namedthm*}

\begin{namedthm*}{Bernoulli's Inequality}
    If $x>-1,$ then
$$
(1+x)^{n} \geq 1+n x, \quad \forall n \in \mathbb{N}
$$
\end{namedthm*}
\begin{namedthm*}{Some useful properties of absolute value}
~\begin{enumerate}
    \item $|a| \leq c \rightarrow -c \leq a \leq c$
    \item $-|a| \leq a \leq |a|$
\end{enumerate}
\end{namedthm*}
\begin{namedthm*}{Definition of means}
  ~
    \begin{enumerate}
        \item The \textit{arithmetic mean} of $a_{1},a_{2},\cdots,a_{n}$ is defined as $A=\frac{a_{1} + a_{2}+\cdots+a_{n}}{n}$.
        \item The \textit{geometric mean} of $a_{1},a_{2},\cdots,a_{n}$ is defined as $G=(a_{1}a_{2} \cdots a_{n})^{\frac{1}{n}}$.
        \item The \textit{harmonic mean} of $a_{1},a_{2},\cdots,a_{n}$ is defined as $H=\frac{n}{\frac{1}{a_1} + \frac{1}{a_2} + \cdots + \frac{1}{a_n}}$.
    \end{enumerate}
\end{namedthm*}



\begin{namedthm*}{The AM-GM-HM Inequality}
Let $A, G, H$ be the arithmetic mean, the geometric mean and
the harmonic mean of the positive numbers $a_{1}, a_{2}, \ldots, a_{n}$ respectively. Then
$$
H \leq G \leq A .
$$
Equality holds if and only if $a_{1}=a_{2}=\cdots=a_{n} .$
\end{namedthm*}

\begin{namedthm*}{Example for finding a set given an inequality}
Solve $|x|+|x+1|<2$\\
\textbf{Solution}: Note that $x$ and $x+1$ change signs at 0 and -1.\\
\textbf{Case} $1 : x \leq-1$\\
In this case, $|x| + |x+1|=-x+(-x-1)=-2 x-1<2,$ so that $2 x>-3$ and $x>-3 / 2 .$ Thus
the points in $(-3 / 2, \infty) \cap(-\infty,-1]=(-3 / 2,-1]$ satisfy the inequality.
\\\textbf{Case} $2 :-1<x<0$\\
In this case, $|x|+|x+1|=-x+(x+1)=1<2$ which is always true. So all the points in $(-1,0)$
satisfy the inequality.\\
\textbf{Case} $3 : x \geq 0$\\
In this case, $|x|+|x+1|=x+(x+1)=2 x+1<2,$ so that $2 x<1$ and $x<1 / 2 .$ Thus the points
in $(-\infty, 1 / 2) \cap[0, \infty)=[0,1 / 2)$ satisfy the inequality.\\
So the solution set is $(-3 / 2,-1] \cup(-1,0) \cup[0,1 / 2)=(-3 / 2,1 / 2)$.
\end{namedthm*}

\begin{namedthm*}{Triangle Inequality}
  For $a, b \in \mathbb{R},|a+b| \leq|a|+|b|$
\end{namedthm*}

\begin{namedthm*}{Corollary 1.8.2}
   $$| | a|-| b| | \leq|a-b|$$
  $$|a-b| \leq|a|+|b|$$
\end{namedthm*}

\begin{namedthm*}{Proving supremums}
     Let L be the supposed supremum. First prove that L is an upper bound. Then prove that L is the smallest upper bound.
\end{namedthm*}

\begin{namedthm*}{Example for proving supremums}
   Let $S$ be a nonempty subset of $\mathbb{R}$ and $a \in \mathbb{R} .$ Let
$$
a+S=\{a+x : x \in S\}.
$$
Prove that if $S$ is bounded above, then $\sup (a+S)=a+\sup S$.

\textbf{Solution:}
\begin{equation}
a+x \leq a+\sup S \quad \forall x \in S
\end{equation}
This says that $a+\sup S$ is an upper bound for $a+S$. Next suppose $v$ is any upper bound of $a+S .$ Then
\begin{equation}
a+x \leq v, \quad \forall x \in S
\end{equation}
\begin{equation}
x \leq v-a \quad \forall x \in S
\end{equation}
So v-a is an upper bound for S. Thus
\begin{equation}
\sup S \leq v-a
\end{equation}
\begin{equation}
a+\sup S \leq v
\end{equation}
We have shown that $a+\sup S$ is an upper bound for $a+S$ and is less than or equal to any other
upper bound for $a+S .$ Thus $\sup (a+S)=a+\sup S .~ \square$
\end{namedthm*}

\begin{namedthm*}{Lemma 1.9.1}
 Let u be an upper bound of $S \subseteq \mathbb{R}$ . Then $u=\sup S$ if and only if $\forall \varepsilon>0,$
$\exists x_{\varepsilon} \in S$ such that $u-\varepsilon<x_{\varepsilon}$ . (The infimum version of this statement was proven in Homework 2)
\end{namedthm*}

\begin{namedthm*}{Supremum property of $\mathbb{R}$}
  Every nonempty subset of $\mathbb{R}$ which is bounded above has a supremum.
\end{namedthm*}

\begin{namedthm*}{Archimedean Property}
   If $x \in \mathbb{R},$ then $\exists n_{x} \in \mathbb{N}$ such that $x<n_{x}$
\end{namedthm*}

\begin{namedthm*}{Corollary 1.9.2}
  For any $\varepsilon>0, \exists n \in \mathbb{N}$ such that
$$
\frac{1}{n}<\varepsilon
$$
\end{namedthm*}

\begin{namedthm*}{Corollary 1.9.3}
 If $x>0,$ then $\exists n \in \mathbb{N}$ such that
$$
n-1 \leq x<n
$$
\end{namedthm*}
\begin{namedthm*}{Theorem 1.11.1}
   Let $a>0$ and $n \in \mathbb{N}$ . There exists a unique positive real number $u$ with
$$
u^{n}=a \text { . }
$$
We call the number u the positive nth root of a and write $u=\sqrt[n]{a}$ or $a^{1 / n}$ .
\end{namedthm*}

\begin{namedthm*}{Result}
  If $a>0$ and $n, m \in \mathbb{N},$ then
$$
\left(a^{1 / n}\right)^{m}=\left(a^{m}\right)^{1 / n}
$$
\end{namedthm*}

\begin{namedthm*}{Theorem 1.11.2 (Properties of rational exponents)}
   ~
    \begin{enumerate}
        \item If $a>0$ and $r, s \in \mathbb{Q},$ then $a^{r+s}=a^{r} a^{s}$ and $\left(a^{r}\right)^{s}=a^{r s}$
        \item If $0<a<b$ and $r \in \mathbb{Q}$ with $r>0,$ then $a^{r}<b^{r}$
        \item If $a>1, r, s \in \mathbb{Q}$ with $r<s,$ then $a^{r}<a^{s}$
    \end{enumerate}
\end{namedthm*}

\begin{namedthm*}{Density Theorem of $\mathbb{Q}$}
  If $a, b \in \mathbb{R}$ is such that $a<b$, then there exists $r \in \mathbb{Q}$ such that
$a<r<b .$
\end{namedthm*}

\begin{namedthm*}{Corollary 1.12.1}
   If $a, b \in \mathbb{R}$ is such that $a<b,$ then there exists an irrational number $x$
such that $a<x<b .$
\end{namedthm*}

\begin{namedthm*}{Corollary 1.12.2}
  Every interval $I \subseteq \mathbb{R}$ contains infinitely many rational numbers and infinitely many irrational numbers.
\end{namedthm*}
\section{Sequences}
\begin{namedthm*}{Theorem 2.1.1}
   If $\left(x_{n}\right)$ converges, then it has exactly one limit.
\end{namedthm*}

\begin{namedthm*}{Theorem 2.2.1}
   Every convergent sequence is bounded.
\end{namedthm*}

\begin{namedthm*}{Theorem 2.2.2}If $\lim _{n \rightarrow \infty} x_{n}=x$ and $\lim _{n \rightarrow \infty} y_{n}=y,$ then
     ~
    \begin{enumerate}
        \item $\lim _{n \rightarrow \infty}\left(x_{n}+y_{n}\right)=x+y$
        \item $\lim _{n \rightarrow \infty}\left(x_{n}-y_{n}\right)=x-y$
        \item $\lim _{n \rightarrow \infty}\left(x_{n} y_{n}\right)=x y$
        \item $\lim _{n \rightarrow \infty}\left(\frac{x_{n}}{y_{n}}\right)=\frac{x}{y},$ provided $y_{n} \neq 0, \forall n \in \mathbb{N},$ and $y \neq 0$
    \end{enumerate}
\end{namedthm*}

\begin{namedthm*}{Corollary 2.2.3}
    If $\left(x_{n}\right)$ converges and $k \in \mathbb{N},$ then
$$
\lim _{n \rightarrow \infty} x_{n}^{k}=\left(\lim _{n \rightarrow \infty} x_{n}\right)^{k}
$$
\end{namedthm*}

\begin{namedthm*}{Classic limit}
   $\lim _{n \rightarrow \infty} \frac{\sin n}{n}=0$
\end{namedthm*}

\begin{namedthm*}{Theorem 2.2.4}
   If $|x_{n}| \rightarrow 0$, then $x_{n} \rightarrow 0.$
\end{namedthm*}
\begin{namedthm*}{Theorem 2.2.5}
   If $0 < b < 1$, then $\lim _{n \rightarrow \infty} b^{n}=0$
\end{namedthm*}

\begin{namedthm*}{Remark on Theorem 2.2.4 and 2.2.5}
   Theorems 2.2.4 and 2.2.5 together imply that $b^n \rightarrow 0$ for all $b$ with $|b| < 1$
\end{namedthm*}


\begin{namedthm*}{Theorem 2.2.6}
    If $c > 0$, then $\lim_{n \rightarrow \infty}c^{\frac{1}{n}}=1$
\end{namedthm*}

\begin{namedthm*}{Theorem 2.2.7}
    ~
    \begin{enumerate}
        \item If $\lim_{n \rightarrow \infty}{x_{n}} = x$, then $\lim_{n \rightarrow \infty}{|x_{n}|} = |x|$ 
        \item If all $x_{n} \geq 0$ and $\lim_{n \rightarrow \infty}{x_{n}} = x$, then $\lim_{n \rightarrow \infty}{\sqrt{x_{n}}} = \sqrt{x}$  
    \end{enumerate}
\end{namedthm*}

\begin{namedthm*}{Theorem 2.2.8}
  $$\lim_{n \rightarrow \infty}{n^{\frac{1}{n}}}=1$$
\end{namedthm*}


\begin{namedthm*}{Theorem 2.2.9}
    ~
    \begin{enumerate}
        \item If $x_{n} \geq 0$ for all $n \in \mathbb{N}$ and $(x_n)$ converges, then $\lim_{n \rightarrow \infty}{x_{n}} \geq 0$
        \item If $(x_{n})$ and $(y_{n})$ are convergent and $x_{n} \geq y_{n}$ for all $n \in \mathbb{N}$, then $$\lim_{n \rightarrow \infty}{x_{n}} \geq \lim_{n \rightarrow \infty}{y_{n}}$$
        \item If $a,b \in \mathbb{R}$ and $a \leq x_{n} \leq b$ for all n and $(x_n)$ is convergent, then $$ a \leq \lim_{n \rightarrow \infty}{x_{n}} \leq b$$
    \end{enumerate}
\end{namedthm*}

\section{Past Midterm Questions}
\begin{namedthm*}{18/19 Question 5}
If $\left(x_{n}\right)$ converges to 0 and $\left(y_{n}\right)$ is bounded, then $\left(x_{n} y_{n}\right)$ converges to $0 .$\\
\textbf{Solution:} Since $\left(y_{n}\right)$ is bounded, there exists $M>0$ such that $\left|y_{n}\right| \leq M$ for all $n \in \mathbb{N} .$ Now
let $\varepsilon>0 .$ since $x_{n} \rightarrow 0$ , there exists $K \in \mathbb{N}$ such that
\[
\left|x_{n}-0\right|<\frac{\varepsilon}{M} \quad \forall n \geq K
\]
Then
\[
n \geq K \Longrightarrow\left|x_{n} y_{n}-0\right|=\left|x_{n} \| y_{n}\right| \leq \frac{\varepsilon}{M} \cdot M=\varepsilon
\]
\end{namedthm*}

\begin{namedthm*}{18/19 Question 6}
    Suppose $(a_n)$ is convergent and $a = \lim_{n \rightarrow \infty}{a_n}$. For each $n \in \mathbb{N}$, let 
    $$b_{n} = \frac{1}{n^2}\sum_{j=1}^{n}(n-j+1)a_{j} = \frac{na_{1}+(n-1)a_{2}+\cdots+2a_{n-1}+a_{n}}{n^2}$$.
    \begin{enumerate}
        \item Prove that for each $n \in \mathbb{N}$,$$b_{n} = \frac{1}{n^2}\sum_{j=1}^{n}(n-j+1)(a_{j}-a) + \frac{n+1}{2n}a$$.\\Proof skipped.
        \item Prove that $(b_{n})$ converges.\\
        \textbf{Solution.} Let $c_{n} = \frac{1}{n^2}\sum_{j=1}^{n}(n-j+1)(a_{j}-a)$ such that $b_{n} = c_{n} + \frac{n+1}{2n}a$. We want to prove that $\lim_{n \rightarrow \infty}{c_n} = 0$.
        \\Let $\epsilon > 0$. Since $\lim_{n \rightarrow \infty}{a_n}=a$, $\exists K_1 \in \mathbb{N}$ s.t. 
        $$n \geq K_{1} \rightarrow |a_{n}-a| < \frac{\epsilon}{2}$$.
        \\Let $C = \sum_{j=1}^{K_1}|a_j-a|$. By A.P $\exists K_2 \in \mathbb{N}$ s.t. $K_2 > \frac{2C}{\epsilon}$. Let $K = \max(K_1,K_2)$. Then for $n \geq K$, we have
        \begin{align*} 
|c_n - 0| &=  \left|\frac{1}{n^2}\sum_{j=1}^{K_1}(n-j+1)(a_{j}-a)\right| \\ 
& \leq  \frac{1}{n^2}\sum_{j=1}^{K_1}(n-j+1)|a_{j}-a|\\ & ~~~~~+ \frac{1}{n^2}\sum_{j=K_{1}+ 1}^{n}(n-j+1)|a_{j}-a|\\
& \leq \frac{1}{n^2}\sum_{j=1}^{K_1}n|a_{j}-a| + \frac{1}{n^2}\sum_{j=K_{1}+ 1}^{n}n\cdot \frac{\epsilon}{2}\\
&\leq \frac{C}{n} + \frac{1}{n^2}(n-K_{1})n\frac{\epsilon}{2}\\
& \leq \frac{C}{K} + \frac{\epsilon}{2}\\
& < \frac{C}{2C/\epsilon} + \frac{\epsilon}{2} = \epsilon
\end{align*}
    \end{enumerate}
\end{namedthm*}

\begin{namedthm*}{13/14 Question 6}
Let $(a_n)$ and $(b_n)$ be defined by setting $$a_1 = 3, b_1 = 2, a_{n+1} = a_n+ 2b_n \text{ and } b_{n+1}= a_n + b_n \text{ for } n \in \mathbb{N}$$. Moreover, let $c_n = \frac{a_n}{b_n}$.
\begin{enumerate}
    \item Express $c_{n+1}$ in terms of $c_{n}$.
    \\\textbf{Solution.} \begin{align*}
        c_{n+1}&=\frac{a_{n+1}}{b_{n+1}}
        \\&=\frac{a_n+2b_{n}}{a_{n}+b_{n}}
        \\&=\frac{\frac{a_n}{b_n} + 2}{\frac{a_n}{b_n} + 1}
        \\&=\frac{c_n + 2}{c_n+1}
    \end{align*}
    \item Prove that $\left|c_{n+1}-\sqrt{2}\right| < r\left|c_{n}-\sqrt{2}\right|$, $r= \sqrt{2}-1$.
    \\\textbf{Solution.} \begin{align*}
        \left|c_{n+1}-\sqrt{2}\right|&=\left|\frac{c_n + 2}{c_n + 1}-\sqrt{2}\right|
        \\&=\frac{1}{c_n + 1}\left|\left(1-\sqrt{2}\right)c_{n} + \left(2- \sqrt{2}\right)\right|
        \\&=\frac{1}{c_n + 1}\left|\left(1-\sqrt{2}\right)\left(c_n- \sqrt{2}\right)\right|
        \\&=\frac{\sqrt{2}-1}{c_n + 1}\left|c_{n} - \sqrt{2}\right|
        \\&=\frac{r}{c_n + 1}\left|c_{n} - \sqrt{2}\right|
        \\& < r\left|c_{n} - \sqrt{2}\right|
    \end{align*}
    \item Prove that $c_{n}$ converges and find its limit.
    \\\textbf{Solution.} For $n \geq 2$ we have by (ii),
    $$\left|c_n-\sqrt{2} \right|< r\left|c_{n-1}-\sqrt{2} \right|<\cdots<r^{n-1}\left|c_{1}-\sqrt{2} \right|$$
    Since $0 < r < 1$, $e^{n-1} \rightarrow 0$. So $$r^{n-1}\left|c_{1}-\sqrt{2} \right| \rightarrow 0.$$ By the Squeeze Theorem, $\left|c_{n}-\sqrt{2} \right| \rightarrow 0$. It follows that $\lim_{n \rightarrow \infty}{c_n} = \sqrt{2}$.
\end{enumerate}
\end{namedthm*}
\section{Tutorial Questions}
\begin{namedthm*}{Ratio theorem from Tutorial 4}
If $$\lim_{n \rightarrow \infty}{\frac{x_{n+1}}{x_n}}=L$$ where $L < 1$, then $$\lim_{n \rightarrow \infty}{x_n} = 0$$.
\end{namedthm*}
\begin{namedthm*}{Expansion of $x^n - y^n$}
$$x^n - y^n = (x-y)(x^{p-1}+x^{p-2}y+x^{p-3}y^{2}+ \cdots +xy^{p-2}+y^{p-1})$$
\end{namedthm*}

\begin{namedthm*}{Tutorial 3 Question 2d}
Prove $$\lim_{n \rightarrow \infty}{\frac{4^n}{n!}} = 0$$.\\
\textbf{Solution.} Notice that 
$$\frac{4^n}{n!} = \frac{4}{1}\cdot\frac{4}{2}\cdot\frac{4}{3}\cdot\underbrace{\frac{4}{4}\cdot\frac{4}{5}\cdots\cdot\frac{4}{n-1}}_\text{$\leq 1$}\cdot\frac{4}{n}$$ when $n \geq 5$. So $$\frac{4^n}{n!} \leq \frac{256}{6n} < \frac{43}{n}$$. Let $\epsilon > 0$. By A.P, $\exists K \in \mathbb{N}$ s.t. $K > \max(5,\frac{43}{\epsilon})$...
\end{namedthm*}
\begin{namedthm*}{Result from Tutorial 1 Question 7}
$$\max(a,b) = \frac{1}{2}\left(a + b + |a-b|\right)$$
$$\min(a,b) = \frac{1}{2}\left(a + b - |a-b|\right)$$
\end{namedthm*}

\begin{namedthm*}{Tutorial 1 Question 3}
For any $n \in \mathbb{N}, n > 1$,$$\left(1+\frac{1}{n-1}\right)^{n-1}<\left(1+\frac{1}{n}\right)^{n}$$
\end{namedthm*}

\begin{namedthm*}{Tutorial 4 Question 1g}
\includegraphics[scale=0.3]{images/tut4-1g.png}
\end{namedthm*}

\begin{namedthm*}{Homework 2 Question 2}
Let $S=\left\{\frac{m}{2 n+3 m} : n, m \in \mathbb{N}\right\} .$ Prove that $\sup S=1 / 3$ and inf $S=0$
\\\textbf{Solution.} First show that $\frac{1}{3}$ is an upper bound. Then, let $\varepsilon>0 .$ By the Archimedean Property, there is a natural number $m_{0}$ such that $m_{0}>\frac{1}{\varepsilon} .$ Let
$x_{\varepsilon}=\frac{m_{0}}{2+3 m_{0}} .$ Then
\[
x_{\varepsilon} \in S \quad \text { and } \quad x_{\varepsilon}=\frac{m_{0}}{2+3 m_{0}}=\frac{1}{3}-\frac{2}{3\left(2+3 m_{0}\right)}>\frac{1}{3}-\frac{1}{m_{0}}>\frac{1}{3}-\varepsilon
\]
By Lemma $1.9 .1,$ sup $S=\frac{1}{3} .$\\
On the other hand, 0 is clearly a lower bound of $S$ . Let $\varepsilon>0 .$ By the Archimedean Property, there
is a natural number $n_{0}$ such that $n_{0}>\frac{1}{\varepsilon} .$ Let $y_{\varepsilon}=\frac{1}{2 n_{0}+3} .$ Then
\[
y_{\varepsilon} \in S \quad \text { and } \quad y_{\varepsilon}=\frac{1}{2 n_{0}+3}<\frac{1}{n_{0}}<0+\varepsilon
\]
By the result of Question $\mathrm{H} 2,$ inf $S=0$
\end{namedthm*}

\begin{namedthm*}{Results from Tutorial 2}~
\begin{enumerate}
    \item If $0<a<b,$ then $a^{n}<b^{n}$ for every $n \in \mathbb{N}$
    \item If $0<a<b$ and $r \in \mathbb{Q}$ with $r>0,$ then $a^{r}<b^{r}$
    \item If $A$ and $B$ are bounded nonempty subsets of $\mathbb{R}$ then $\sup (A \cup B)=\max (\sup A, \sup B)$
\end{enumerate}
\end{namedthm*}

\begin{namedthm*}{Tutorial 3 Question 7} 
If $\lim _{n \rightarrow \infty} x_{n}=x$ , then
\[
\lim _{n \rightarrow \infty} \frac{x_{1}+x_{2}+\cdots+x_{n}}{n}=x
\]
\end{namedthm*}



\end{multicols}

\end{document}