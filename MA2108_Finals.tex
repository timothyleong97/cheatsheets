% Credits to Ning Yuan for the format

\documentclass[10pt,landscape]{article}
\usepackage{amssymb,amsmath,amsthm,amsfonts,bm}
\usepackage{multicol,multirow}
\usepackage{calc}
\usepackage{ifthen}
\usepackage[landscape]{geometry}
\usepackage[colorlinks=true,citecolor=blue,linkcolor=blue]{hyperref}
\usepackage{graphicx}
\graphicspath{ {./images/} }
\ifthenelse{\lengthtest { \paperwidth = 11in}}
    { \geometry{top=.2in,left=.2in,right=.2in,bottom=.2in} }
	{\ifthenelse{ \lengthtest{ \paperwidth = 297mm}}
		{\geometry{top=1cm,left=1cm,right=1cm,bottom=1cm} }
		{\geometry{top=1cm,left=1cm,right=1cm,bottom=1cm} }
	}
\pagestyle{empty}
\makeatletter
\renewcommand{\section}{\@startsection{section}{1}{0mm}%
                                {-1ex plus -.5ex minus -.2ex}%
                                {0.5ex plus .2ex}%x
                                {\normalfont\large\bfseries}}
\renewcommand{\subsection}{\@startsection{subsection}{2}{0mm}%
                                {-1explus -.5ex minus -.2ex}%
                                {0.5ex plus .2ex}%
                                {\normalfont\normalsize\bfseries}}
\renewcommand{\subsubsection}{\@startsection{subsubsection}{3}{0mm}%
                                {-1ex plus -.5ex minus -.2ex}%
                                {1ex plus .2ex}%
                                {\normalfont\small\bfseries}}
\makeatother
\setcounter{secnumdepth}{0}
\setlength{\parindent}{0pt}
\setlength{\parskip}{0pt plus 0.5ex}

\newcommand{\matr}[1]{\bm{#1}}
\newcommand{\vect}[1]{\bm{#1}}
\newcommand{\adj}{\operatorname{\textbf{adj}}}
\newcommand{\lspan}{\operatorname{span}}
\newcommand{\rank}{\operatorname{rank}}
\newcommand{\nullity}{\operatorname{nullity}}
\newcommand{\Ker}{\operatorname{Ker}}
\newcommand{\norm}[1]{\left\lVert#1\right\rVert}

\theoremstyle{definition}
\newcommand{\thistheoremname}{}
\newtheorem*{genericthm*}{\thistheoremname}
\newenvironment{namedthm*}[1]
{\renewcommand{\thistheoremname}{#1}\begin{genericthm*}}
{\end{genericthm*}}

% -----------------------------------------------------------------------

\title{MA2108 Cheatsheet 19/20 Sem 1 Finals}

\begin{document}

\begin{center}
	{\large MA2108 Cheatsheet 19/20 Sem 1 Finals}
\end{center}

\raggedright
\footnotesize

\begin{multicols}{3}

	\setlength{\premulticols}{1pt}
	\setlength{\postmulticols}{1pt}
	\setlength{\multicolsep}{1pt}
	\setlength{\columnsep}{2pt}

	% \section{The Real Numbers}

	\begin{namedthm*}{Result 1}
		\(2^{n-1} \leq n!~ \text{and} ~ n < 2^{n}~\forall n \in \mathbb{N}\)
	\end{namedthm*}

	\begin{namedthm*}{Result 2}
		\(n^{2} \leq 2^{n}~\forall n \in \mathbb{N} \backslash \{2,3,4\}\)
	\end{namedthm*}

	\begin{namedthm*}{Well-ordering principle}
		Every nonempty subset \(A\) of \(\mathbb{N}\) has \(a\) least (first) element,
		i.e. there exists \(p \in A\) such that \(p \leq a\) for all \(a \in A\)
	\end{namedthm*}

	\begin{namedthm*}{Lemma 1.5.1}
		~
		\begin{enumerate}
			\item If \(c > 1\), then \(c^{n} > c\) for every natural number \(n \geq 2\).
			\item If \(0 < c < 1\), then \(c^{n} < c\) for every natural number \(n \geq 2\).
		\end{enumerate}
	\end{namedthm*}

	\begin{namedthm*}{Theorem 1.5.3}
		If \(a \in \mathbb{R}\) is such that \(0 \leq a< \varepsilon\) for every positive number \(\varepsilon,\) then \(a=0\)
	\end{namedthm*}

	\begin{namedthm*}{Bernoulli's Inequality}
		If \(x>-1,\) then
		$
			(1+x)^{n} \geq 1+n x, \quad \forall n \in \mathbb{N}
		$
	\end{namedthm*}

	\begin{namedthm*}{Harmonic mean}
		The \textit{harmonic mean} of \(a_{1},a_{2},\cdots,a_{n}\) is defined as \(H=\frac{n}{\frac{1}{a_1} + \frac{1}{a_2} + \cdots + \frac{1}{a_n}}\).
	\end{namedthm*}


	\begin{namedthm*}{The AM-GM-HM Inequality}
		\(H \leq G \leq A .\)
		Equality holds if and only if \(a_{1}=a_{2}=\cdots=a_{n} .\)
	\end{namedthm*}


	\begin{namedthm*}{Triangle Inequality}
		For \(a, b \in \mathbb{R},|a+b| \leq|a|+|b|\)
	\end{namedthm*}

	\begin{namedthm*}{Corollary 1.8.2}
		$$| | a|-| b| | \leq|a-b|$$
		$$|a-b| \leq|a|+|b|$$
	\end{namedthm*}



	\begin{namedthm*}{Lemma 1.9.1}
		Let u be an upper bound of \(S \subseteq \mathbb{R}\) . Then \(u=\sup S\) if and only if \(\forall \varepsilon>0,\)
		\(\exists x_{\varepsilon} \in S\) such that \(u-\varepsilon<x_{\varepsilon}\) . (The infimum version of this statement was proven in Homework 2)
	\end{namedthm*}

	\begin{namedthm*}{Supremum property of \(\mathbb{R}\)}
		Every nonempty subset of \(\mathbb{R}\) which is bounded above has a supremum.
	\end{namedthm*}

	\begin{namedthm*}{Archimedean Property}
		If \(x \in \mathbb{R},\) then \(\exists n_{x} \in \mathbb{N}\) such that \(x<n_{x}\)
	\end{namedthm*}

	\begin{namedthm*}{Corollary 1.9.2}
		For any \(\varepsilon>0, \exists n \in \mathbb{N}\) such that
		$
			\frac{1}{n}<\varepsilon
		$
	\end{namedthm*}

	\begin{namedthm*}{Corollary 1.9.3}
		If \(x>0,\) then \(\exists n \in \mathbb{N}\) such that
		$
			n-1 \leq x<n
		$
	\end{namedthm*}
	\begin{namedthm*}{Theorem 1.11.1}
		Let \(a>0\) and \(n \in \mathbb{N}\) . There exists a unique positive real number \(u\) with
		$
			u^{n}=a \text { . }
		$
		We call the number u the positive nth root of a and write \(u=\sqrt[n]{a}\) or \(a^{1 / n}\) .
	\end{namedthm*}

	\begin{namedthm*}{Result}
		If \(a>0\) and \(n, m \in \mathbb{N},\) then
		$$
			\left(a^{1 / n}\right)^{m}=\left(a^{m}\right)^{1 / n}
		$$
	\end{namedthm*}

	\begin{namedthm*}{Density Theorem of \(\mathbb{Q}\)}
		If \(a, b \in \mathbb{R}\) is such that \(a<b\), then there exists \(r \in \mathbb{Q}\) such that
		\(a<r<b .\)
	\end{namedthm*}

	\begin{namedthm*}{Corollary 1.12.1}
		If \(a, b \in \mathbb{R}\) is such that \(a<b,\) then there exists an irrational number \(x\)
		such that \(a<x<b .\)
	\end{namedthm*}

	\begin{namedthm*}{Corollary 1.12.2}
		Every interval \(I \subseteq \mathbb{R}\) contains infinitely many rational numbers and infinitely many irrational numbers.
	\end{namedthm*}

	% \section{Sequences}

	\begin{namedthm*}{Theorem 2.1.1}
		If \(\left(x_{n}\right)\) converges, then it has exactly one limit.
	\end{namedthm*}

	\begin{namedthm*}{Theorem 2.2.1}
		Every convergent sequence is bounded.
	\end{namedthm*}

	\begin{namedthm*}{Theorem 2.2.2}If \(\lim _{n \rightarrow \infty} x_{n}=x\) and \(\lim _{n \rightarrow \infty} y_{n}=y,\) then
		~
		\begin{enumerate}
			\item \(\lim _{n \rightarrow \infty}\left(x_{n}+y_{n}\right)=x+y\)
			\item \(\lim _{n \rightarrow \infty}\left(x_{n}-y_{n}\right)=x-y\)
			\item \(\lim _{n \rightarrow \infty}\left(x_{n} y_{n}\right)=x y\)
			\item \(\lim _{n \rightarrow \infty}\left(\frac{x_{n}}{y_{n}}\right)=\frac{x}{y},\) provided \(y_{n} \neq 0, \forall n \in \mathbb{N},\) and \(y \neq 0\)
		\end{enumerate}
	\end{namedthm*}

	\begin{namedthm*}{Corollary 2.2.3}
		If \(\left(x_{n}\right)\) converges and \(k \in \mathbb{N},\) then
		$$
			\lim _{n \rightarrow \infty} x_{n}^{k}=\left(\lim _{n \rightarrow \infty} x_{n}\right)^{k}
		$$
	\end{namedthm*}

	\begin{namedthm*}{Classic limit}
		\(\lim _{n \rightarrow \infty} \frac{\sin n}{n}=0\)
	\end{namedthm*}

	\begin{namedthm*}{Theorem 2.2.4}
		If \(|x_{n}| \rightarrow 0\), then \(x_{n} \rightarrow 0.\)
	\end{namedthm*}
	\begin{namedthm*}{Theorem 2.2.5}
		If \(0 < b < 1\), then \(\lim _{n \rightarrow \infty} b^{n}=0\)
	\end{namedthm*}

	\begin{namedthm*}{Remark on Theorem 2.2.4 and 2.2.5}
		Theorems 2.2.4 and 2.2.5 together imply that \(b^n \rightarrow 0\) for all \(b\) with \(|b| < 1\)
	\end{namedthm*}


	\begin{namedthm*}{Theorem 2.2.6}
		If \(c > 0\), then \(\lim_{n \rightarrow \infty}c^{\frac{1}{n}}=1\)
	\end{namedthm*}

	\begin{namedthm*}{Theorem 2.2.7}
		~
		\begin{enumerate}
			\item If \(\lim_{n \rightarrow \infty}{x_{n}} = x\), then \(\lim_{n \rightarrow \infty}{|x_{n}|} = |x|\)
			\item If all \(x_{n} \geq 0\) and \(\lim_{n \rightarrow \infty}{x_{n}} = x\), then \(\lim_{n \rightarrow \infty}{\sqrt{x_{n}}} = \sqrt{x}\)
		\end{enumerate}
	\end{namedthm*}

	\begin{namedthm*}{Theorem 2.2.8}
		\(\lim_{n \rightarrow \infty}{n^{\frac{1}{n}}}=1\)
	\end{namedthm*}


	\begin{namedthm*}{Theorem 2.2.9}
		~
		\begin{enumerate}
			\item If \(x_{n} \geq 0\) for all \(n \in \mathbb{N}\) and \((x_n)\) converges, then \(\lim_{n \rightarrow \infty}{x_{n}} \geq 0\)
			\item If \((x_{n})\) and \((y_{n})\) are convergent and \(x_{n} \geq y_{n}\) for all \(n \in \mathbb{N}\), then \(\lim_{n \rightarrow \infty}{x_{n}} \geq \lim_{n \rightarrow \infty}{y_{n}}\)
			\item If \(a,b \in \mathbb{R}\) and \(a \leq x_{n} \leq b\) for all n and \((x_n)\) is convergent, then \( a \leq \lim_{n \rightarrow \infty}{x_{n}} \leq b\)
		\end{enumerate}
	\end{namedthm*}

	\begin{namedthm*}{Monotone Convergence Theorem}
		If \(\left(x_{n}\right)\) is monotone and bounded, then it converges. \(\lim _{n \rightarrow \infty} x_{n}=\left\{\begin{array}{ll}{\sup \left\{x_{n}: n \in \mathbb{N}\right\}} & {\text { if } x_{n} \uparrow} \\ {\inf \left\{x_{n}: n \in \mathbb{N}\right\}} & {\text { if } x_{n} \downarrow}\end{array}\right.\)
	\end{namedthm*}

	\begin{namedthm*}{Nested Interval Theorem}
		Let \(I_{n}=\left[a_{n}, b_{n}\right], n \in \mathbb{N}\) be a nested sequence of closed bounded intervals, that is, \(I_{n} \supseteq I_{n+1}\)
		for \(n \in \mathbb{N} .\) Then the intersection \(\bigcap_{n=1}^{\infty} I_{n}=\left\{x: x \in I_{n} \forall n \in \mathbb{N}\right\}\) is nonempty. In addition, if length of \(I_{n}=b_{n}-a_{n} \rightarrow 0\) then\(\bigcap_{n=1}^{\infty} I_{n}\) contains exactly one point.
	\end{namedthm*}

	\begin{namedthm*}{Theorem 2.4.1}
		If \(\left(x_{n}\right)\) converges to \(x,\) then any subsequence \(\left(x_{n_{k}}\right)\) also converges to \(x\).
	\end{namedthm*}

	\begin{namedthm*}{Corollary 2.4.2}
		If \(\left(x_{n}\right)\) has a subsequence which is divergent, then \(\left(x_{n}\right)\) diverges.
	\end{namedthm*}

	\begin{namedthm*}{Corollary 2.4.3}
		If \(\left(x_{n}\right)\) has two convergent subsequences whose limits are not equal, then
		\(\left(x_{n}\right)\) diverges.
	\end{namedthm*}

	\begin{namedthm*}{Monotone Subsequence Theorem}
		Every sequence has a monotone subsequence.
	\end{namedthm*}

	\begin{namedthm*}{Bolzano-Weierstrass Theorem}
		Every bounded sequence has a convergent subsequence.
	\end{namedthm*}

	\begin{namedthm*}{Lemma 2.5.1}
		Let \(x \in \mathbb{R} .\) Then there exists an increasing rational sequence \(\left(r_{n}\right)\) which con-
		verges to \(x .\)
	\end{namedthm*}

	\begin{namedthm*}{Theorem 2.5.3}
		If \(a \geq 1\) and \(\left(r_{n}\right)\) is a decreasing rational sequence with limit \(x,\) then
		\(
		\lim _{n \rightarrow \infty} a^{r_{n}}=a^{x} \text { . }
		\)
	\end{namedthm*}

	\begin{namedthm*}{Theorem 2.5.4 (Properties of exponents)}
		~
		\begin{enumerate}
			\item \(a^{x+y}=a^{x} a^{y}\)
			\item \(\left(a^{x}\right)^{y}=a^{x y}\)
			\item If \(a>1\) and \(x<y,\) then \(a^{x}<a^{y}\)
		\end{enumerate}
	\end{namedthm*}

	\begin{namedthm*}{Theorem 2.6.1}
		Let \(\left(x_{n}\right)\) be a bounded sequence and let \(M=\limsup x_{n}\).
		\begin{enumerate}
			\item For each \(\varepsilon>0,\) there are at most finitely many \(n^{\prime}\) such that \(x_{n} \geq M+\varepsilon .\) Equivalently, there exists \(K \in \mathbb{N}\) such that \(n \geq K \Longrightarrow x_{n}<M+\varepsilon\).
			\item For each \(\varepsilon>0,\) there are infinitely many \(n^{\prime} s\) such that \(x_{n}>M-\varepsilon\).
		\end{enumerate}
	\end{namedthm*}

	\begin{namedthm*}{Theorem 2.6.2}
		Let \(\left(x_{n}\right)\) be a bounded sequence and let \(m=\liminf x_{n}\).
		\begin{enumerate}
			\item For each \(\varepsilon>0,\) there are at most only finitely many \(n\) 's such that \(x_{n} \leq m-\varepsilon .\) Equivalently, there exists \(K \in \mathbb{N}\) such that \(n \geq K \Longrightarrow x_{n}>m-\varepsilon\).
			\item For each \(\varepsilon>0,\) there are infinitely many \(n^{\prime}\) s such that \(x_{n}<m+\varepsilon .\)
		\end{enumerate}
	\end{namedthm*}

	\begin{namedthm*}{Theorem 2.6.3}
		Let \(\left(x_{n}\right)\) be a bounded sequence. Then \(\left(x_{n}\right)\) converges if and only if
		\(
		\lim \sup x_{n}=\lim \inf x_{n}
		\).
	\end{namedthm*}


	\begin{namedthm*}{Theorem 2.6.4}
		Let \(\left(x_{n}\right)\) and \(\left(y_{n}\right)\) be bounded sequence such that \(x_{n} \leq y_{n}\) for every \(n \in \mathbb{N}\). Then \(\limsup x_{n} \leq \lim \sup y_{n}\) and \(\liminf x_{n} \leq \liminf y_{n}\).
	\end{namedthm*}

	\begin{namedthm*}{Definition of a Cauchy sequence}
		A sequence \(\left(x_{n}\right)\) is called a Cauchy sequence if for every \(\varepsilon>0,\) there exists \(K \in \mathbb{N}\) such that \(\left|x_{n}-x_{m}\right|<\varepsilon, \quad \forall n, m \geq K\).
	\end{namedthm*}

	\begin{namedthm*}{Theorem 2.7.1}
		Every convergent sequence is Cauchy.
	\end{namedthm*}

	\begin{namedthm*}{Cauchy criterion}
		Every Cauchy sequence is convergent (and thus bounded).
	\end{namedthm*}

	\begin{namedthm*}{Contractive sequences}
		A sequence \(\left(x_{n}\right)\) is said to be contractive if \(\exists C\) with \(0<C<1\) such that \(\left|x_{n+2}-x_{n+1}\right| \leq C\left|x_{n+1}-x_{n}\right|, \quad \forall n \in \mathbb{N}\).
	\end{namedthm*}

	\begin{namedthm*}{Theorem 2.7.3}
		Every contractive sequence is Cauchy (and so is convergent).
	\end{namedthm*}

	\begin{namedthm*}{Partial fraction}
		\(\frac{1}{k(k+1)}=\frac{1}{k}-\frac{1}{k+1}\)
	\end{namedthm*}

	\begin{namedthm*}{Theorem 3.1.1}
		If the series \(\sum_{n=1}^{\infty} a_{n}\) and \(\sum_{n=1}^{\infty} b_{n}\) are convergent, then the series \(\sum_{n=1}^{\infty}\left(a_{n}+b_{n}\right)=\sum_{n=1}^{\infty} a_{n}+\sum_{n=1}^{\infty} b_{n}\), and \(\sum_{n=1}^{\infty} c a_{n}=c \sum_{n=1}^{\infty} a_{n}\).
	\end{namedthm*}

	\begin{namedthm*}{Theorem 3.1.2}
		If \(\sum_{n=1}^{\infty} a_{n}\) converges, then \(\lim _{n \rightarrow \infty} a_{n}=0\).
	\end{namedthm*}

	\begin{namedthm*}{The n-th term divergence test}
		If \(\lim _{n \rightarrow \infty} a_{n} \neq 0,\) then \(\sum_{n=1}^{\infty} a_{n}\) diverges.
	\end{namedthm*}

	\begin{namedthm*}{Cauchy criterion for series}
		The series \(\sum_{n=1}^{\infty} a_{n}\) converges if and only iffor every \(\varepsilon>0,\) there exists \(K \in \mathbb{N}\) such that \(\left|a_{n+1}+a_{n+2}+\cdots+a_{m}\right|<\varepsilon, \quad \forall m>n \geq K\).
	\end{namedthm*}

	\begin{namedthm*}{Theorem 3.2.1}
		If \(a_{n} \geq 0\) for all \(n,\) then the series \(\sum_{n=1}^{\infty} a_{n}\) converges if and only if the sequence \(\left(s_{n}\right)\) of partial sums is bounded.
	\end{namedthm*}

	\begin{namedthm*}{Theorem 3.2.2}
		If \(p>1,\) then the \(p\) -series \(\sum_{n=1}^{\infty} \frac{1}{n^{p}}\) converges.
	\end{namedthm*}

	\begin{namedthm*}{Theorem 3.2.3}
		If \(0<p \leq 1,\) then the p-series \(\sum_{n=1}^{\infty} \frac{1}{n^{p}}\) diverges.
	\end{namedthm*}

	\begin{namedthm*}{Comparison Test}
		Suppose that \(0 \leq a_n \leq b_n, \quad \forall n \geq K\) for some \(K \in \mathbb{N}\). Then \(\sum_{n=1}^{\infty} b_{n}\) converges \(\Longrightarrow \sum_{n=1}^{\infty} a_{n}\) converges. \(\sum_{n=1}^{\infty} a_{n}\) diverges \(\Longrightarrow \sum_{n=1}^{\infty} b_{n}\) diverges.
	\end{namedthm*}


	\begin{namedthm*}{Limit Comparison Test}
		Let \(\sum_{n=1}^{\infty} a_{n}\) and \(\sum_{n=1}^{\infty} b_{n}\) be series with positive terms. Suppose \(\rho=\lim _{n \rightarrow \infty} \frac{a_{n}}{b_{n}}\) exists. If \(\rho>0,\) then either the two series both converge or both diverge. If \(\rho=0\) and \(\sum_{n=1}^{\infty} b_{n}\) converges, then \(\sum_{n=1}^{\infty} a_{n}\) converges.
	\end{namedthm*}

	\begin{namedthm*}{Alternating Series Test}
		If \(\left(a_{n}\right)\) is a decreasing sequence such that \(a_{n}>0\) for all \(n\) and \(\lim _{n \rightarrow \infty} a_{n}=0,\) then the alternating series \(\sum_{n=1}^{\infty}(-1)^{n+1} a_{n}\) converges.
	\end{namedthm*}

	\begin{namedthm*}{Theorem 3.4.1}
		If the series \(\sum_{n=1}^{\infty} a_{n}\) converges absolutely, then it converges.
	\end{namedthm*}

	\begin{namedthm*}{Ratio Test}
		Suppose that all the terms of the series \(\sum_{n=1}^{\infty} a_{n}\) are nonzero and the limit \(\rho=\lim _{n \rightarrow \infty}\left|\frac{a_{n+1}}{a_{n}}\right|\) exists. If \(\rho<1,\) then the series \(\sum_{n=1}^{\infty} a_{n}\) converges absolutely. If \(\rho>1,\) then the series \(\sum_{n=1}^{\infty} a_{n}\) diverges. No conclusion if \(\rho=1\).
	\end{namedthm*}
	\begin{namedthm*}{Ross pg. 79}
		\(\liminf \left|\frac{s_{n+1}}{s_{n}}\right| \leq \liminf \left|s_{n}\right|^{1 / n} \leq \limsup \left|s_{n}\right|^{1 / n} \leq \limsup \left|\frac{s_{n+1}}{s_{n}}\right|\). \textbf{Proof.} Let \(\alpha=\limsup \left|s_{n}\right|^{1 / n}\) and \(L=\limsup \left|\frac{s_{n+1}}{s_{n}}\right|\). It suffices to show  \(\alpha \leq L_{1}\) for any \(L_{1}>L\). \(\exists K\) such that \(\sup \left\{\left|\frac{s_{n+1}}{s_{n}}\right|: n \geq K\right\}<L_{1}\). Thus \(\left|\frac{s_{n+1}}{s_{n}}\right|<L_{1}\) for \(n \geq K\), and we write \(\left|s_{n}\right|=\left|\frac{s_{n}}{s_{n-1}}\right| \cdot\left|\frac{s_{n-1}}{s_{n-2}}\right| \cdots\left|\frac{s_{K+1}}{s_{K}}\right| \cdot\left|s_{K}\right|\), such that \(\left|s_{n}\right|<L_{1}^{n-K}\left|s_{K}\right|\). Let \(a=L_{1}^{-N}\left|s_{N}\right| > 0\). Then \(\left|s_{n}\right|^{1 / n}<L_{1} a^{1 / n} \rightarrow L_1\) for \(n>K\). So \(\alpha=\limsup \left|s_{n}\right|^{1 / n} \leq L_{1}\).
	\end{namedthm*}

	\begin{namedthm*}{Result from Ch 3 Pg 16}
		If  \(\limsup\left|\frac{a_{n+1}}{a_{n}}\right|<1,\) then the series \(\sum_{n=1}^{\infty} a_{n}\) converges absolutely. If  \(\liminf\left|\frac{a_{n+1}}{a_{n}}\right|>1,\) then the series \(\sum_{n=1}^{\infty} a_{n}\) diverges.
	\end{namedthm*}

	\begin{namedthm*}{Result from Ch 3 Pg 17}
		Let \(\rho=\lim \sup \left|a_{n}\right|^{1 / n}\). If \(\rho<1,\) then the series \(\sum_{n=1}^{\infty} a_{n}\) converges absolutely. If \(\rho>1,\) then the series \(\sum_{n=1}^{\infty} a_{n}\) diverges. No conclusion if \(\rho=1\).
	\end{namedthm*}

	% \begin{namedthm*}{Proof of Pg 17}
	% 	Suppose \(\rho<1,\) and select \(\epsilon>0\) so that \(\rho+\epsilon<1\). For some K, \(\rho-\epsilon<\sup \left\{\left|a_{n}\right|^{1 / n}: n>N\right\}<\rho+\epsilon\). \(\left|a_{n}\right|<(\rho+\epsilon)^{n}\). The right side makes a geometric series, by the Comp Test, \(\sum|a_n|\) converges. If \(\rho>1,\) then a subsequence of \(\left|a_{n}\right|^{1 / n}\) has
	% 	limit \(\rho>1 .\) It follows that \(\left|a_{n}\right|>1\) for infinitely many choices
	% 	of \(n .\) In particular, the sequence \(\left(a_{n}\right)\) cannot converge
	% 	to \(0,\) so the series \(\sum a_{n}\) cannot converge.
	% \end{namedthm*}

	\begin{namedthm*}{Proof of Ch 3 Pg 16}
		\(\liminf \left|\frac{a_{n+1}}{a_{n}}\right| \leq \limsup \left|a_{n}\right|^{1 / n} \leq \limsup \left|\frac{a_{n+1}}{a_{n}}\right|\). Prove using Pg 17.
	\end{namedthm*}

	\begin{namedthm*}{Root Test}
		Suppose \(\rho=\lim _{n \rightarrow \infty}\left|a_{n}\right|^{1 / n}\) exists. If \(\rho<1,\) then the series \(\sum_{n=1}^{\infty} a_{n}\) converges absolutely. If \(\rho>1,\) then the series \(\sum_{n=1}^{\infty} a_{n}\) diverges. No conclusion if \(\rho = 1\).
	\end{namedthm*}

	\begin{namedthm*}{Theorem 3.6.1}
		If the series \(\sum_{n=1}^{\infty} a_{n}\) converges, then any series obtained by grouping the
		terms of \(\sum_{n=1}^{\infty} a_{n}\) also converges and has the same value as \(\sum_{n=1}^{\infty} a_{n} .\)
	\end{namedthm*}

	\begin{namedthm*}{Theorem 3.7.1}
		If the series \(\sum_{n=1}^{\infty} a_{n}\) converges absolutely, then any rearrangement of \(\sum_{n=1}^{\infty} a_{n}\) also converges and has the same sum as \(\sum_{n=1}^{\infty} a_{n} .\)
	\end{namedthm*}

	\begin{namedthm*}{Theorem 3.8.1}
		\(e=\sum_{n=0}^{\infty} \frac{1}{n !}\) and for each \(n \in \mathbb{N}, e-\sum_{j=0}^{n}\frac{1}{j !}<\frac{1}{n(n !)}\)
	\end{namedthm*}

	\begin{namedthm*}{Theorem 3.8.2}
		\(e\) is irrational.
	\end{namedthm*}

	\begin{namedthm*}{Sequential criterion}
		\(\lim _{x \rightarrow a} f(x)=L \Longleftrightarrow \text{If }\left(x_{n}\right)\) is any sequence in the domain of \(f\) such that \(x_{n} \neq a\) for all \(n\) and \(x_{n} \rightarrow a\), then \(f(x_n)\rightarrow L\). Note that L and a can be infinity.
	\end{namedthm*}

	\begin{namedthm*}{Corollary 4.2.2}
		\(\lim _{x \rightarrow a} f(x) \neq L \Longleftrightarrow\) there is a sequence \(\left(x_{n}\right)\) in the domain of \(f\) such that \(x_{n} \neq a\) for
		all n and \(x_{n} \rightarrow a, \text{ but } f\left(x_{n}\right) \not\rightarrow L\)
	\end{namedthm*}

	\begin{namedthm*}{Divergent Criterion}
		\textbf{Method 1}. Find a sequence \(\left(x_{n}\right)\) in the domain of \(f\) such that \(x_{n} \neq a\) for all \(n\) and \(x_{n} \rightarrow a,\) but \(\left(f\left(x_{n}\right)\right)\) diverges. \textbf{Method 2}. Find two sequences \(\left(x_{n}\right)\) and \(\left(y_{n}\right)\) in the domain of \(f\) such that \(x_{n} \neq a\) and \(y_{n} \neq a\) for all \(n\) and \(x_{n} \rightarrow a, y_{n} \rightarrow a,\) but \(\lim _{n \rightarrow \infty} f\left(x_{n}\right) \neq \lim _{n \rightarrow \infty} f\left(y_{n}\right)\).
	\end{namedthm*}

	\begin{namedthm*}{Dirichlet Function}
		\(f(x)=\left\{\begin{array}{cc}{1} & {\text { if } x \in \mathbb{Q}} \\ {0} & {\text { if } x \notin \mathbb{Q}}\end{array}\right.\) is continuous nowhere.
	\end{namedthm*}

	\begin{namedthm*}{Lemma 4.2.3}
		There exists a sequence \(\left(x_{n}\right)\) such that \(x_{n}\) is rational for all \(n, x_{n} \neq c\) for all \(n\) and \(x_{n} \rightarrow c \in \mathbb{R}\), and a sequence \(\left(y_{n}\right)\) such that \(y_{n}\) is irrational for all \(n, y_{n} \neq c\) for all \(n\) and \(y_{n} \rightarrow c\)
	\end{namedthm*}

	\begin{namedthm*}{Theorem 4.3.1}
		Suppose \(f\) is defined in a deleted neighborhood of \(x=a\). If \(\lim _{x \rightarrow a} f(x)\) exists, then \(f\) is
		bounded in a deleted neighborhood of \(x=a,\)that is, \(\exists M>0\) and \(\delta>0\) such that \(0<|x-a|<\delta \Longrightarrow|f(x)| \leq M\).
	\end{namedthm*}

	\begin{namedthm*}{Theorem 4.3.2}
		Limit laws apply to functions.
	\end{namedthm*}

	\begin{namedthm*}{Theorem 4.3.3}
		If \(f(x) \leq g(x)\) for all \(x\) in a deleted neighborhood of \(x=a\) and both \(\lim _{x \rightarrow a} f(x)\) and
		\(\lim _{x \rightarrow a} g(x)\) exist, then \(\lim _{x \rightarrow a} f(x) \leq \lim _{x \rightarrow a} g(x)\).
	\end{namedthm*}

	\begin{namedthm*}{Theorem 4.3.4}
		If \(f\) is defined in a deleted neighborhood of \(x=a\) and \(\lim _{x \rightarrow a} f(x)=L\) exists and \(L>0,\)
		then \(\exists \delta>0\) such that
		\(
		f(x)>0 \quad \forall x \text { such that } 0<|x-a|<\delta
		\)
	\end{namedthm*}

	\begin{namedthm*}{Functions proven to meet seq crit}
		Polynomial, abs, sqrt, \(x^r\),  \(a^x\), sin and cos, rational functions (f(x)/q(x)),
	\end{namedthm*}

	\begin{namedthm*}{Thomae's function}
		Let \(f:(0,1) \rightarrow \mathbb{R}\) be defined by
		\[
			f(x)=\left\{\begin{array}{ll}{0} & {\text { if } x \text { is irrational }} \\ {\frac{1}{q}} & {\text { if } x=\frac{p}{q}, p, q \in \mathbb{N} \text { and } \operatorname{gcd}(p, q)=1}\end{array}\right.
		\]
		which is discontinuous at all rational points and continuous at all irrational points.
	\end{namedthm*}

	\begin{namedthm*}{Theorem 5.2.3}
		Theorem 5.2.3. Suppose that \(f: A \rightarrow \mathbb{R}, g: B \rightarrow \mathbb{R}\) and \(f(A) \subseteq B,\) so that \(g \circ f\) is defined. If \(f\) is
		continuous on \(A,\) and \(g\) is continuous on \(B,\) then \(g \circ f\) is continuous on \(A .\)
	\end{namedthm*}

	\begin{namedthm*}{Remarks on continuous functions.}
		\(\sqrt{\sin x}\) is continuous on \((0,\pi)\).
	\end{namedthm*}

	\begin{namedthm*}{Theorem 5.3.1}
		If \(f\) is continuous on \([a, b],\) then \(f\) is bounded on \([a, b]\)
	\end{namedthm*}

	\begin{namedthm*}{Extreme-value Theorem.}
		If \(f\) is continuous on \([a, b],\) then there exists \(x_{1}, x_{2} \in[a, b]\) such that
		\(
		f\left(x_{1}\right) \leq f(x) \leq f\left(x_{2}\right) \quad \forall x \in[a, b]
		\)
	\end{namedthm*}

	\begin{namedthm*}{Location of Roots Theorem}
		If \(f\) is continuous on \([a, b], f(a)<0<f(b),\) then there exists a point \(c\),
		in \((a, b)\) such that \(f(c)=0 .\)
	\end{namedthm*}

	\begin{namedthm*}{Intermediate Value Theorem}	If \(f\) is continuous on \([a, b],\) and 	\(k\) is between \(f(a)\) and \(f(b),\) 	then there exists a point \(c\) in \((a,	 b)\) such that \(f(c)=k .\)
	\end{namedthm*}

	\begin{namedthm*}{Theorem 5.3.2}
		If \(f\) is continuous on \([a, b],\) then
		\(
		f([a, b])=[m, M],
		\)
		where \(m=\inf f([a, b])\) and \(M=\sup f([a, b])\)
	\end{namedthm*}

	\begin{namedthm*}{Theorem 5.4.1}
		Let I \(\subseteq \mathbb{R}\) be an interval and \(f: I \rightarrow \mathbb{R}\) be an increasing function. If \(c \in I\) is not an
		end point of \(I,\) then \(\lim _{x \rightarrow c^{-}} f(x)\) and \(\lim _{x \rightarrow c^{+}} f(x)\) exist and they are given by
		\(
		\lim _{x \rightarrow c^{-}} f(x)=\sup \{f(x): x \in I, x<c\} \text { and } \lim _{x \rightarrow c^{+}} f(x)=\inf \{f(x): x \in I, x>c\}
		\)
	\end{namedthm*}

	\begin{namedthm*}{Continuous Inverse Theorem}
		Let I \(\subseteq \mathbb{R}\) be an interval and \(f: I \rightarrow \mathbb{R}\) be a strictly monotone
		function. If \(f\) is continuous on \(I\) and \(J=f(I),\) then its inverse function \(f^{-1}: J \rightarrow \mathbb{R}\) is strictly
		monotone and continuous on \(J .\)
	\end{namedthm*}

	\begin{namedthm*}{Uniform Continuity}
		\(x, y \in I,|x-y|<\delta \Longrightarrow|f(x)-f(y)|<\varepsilon\).
	\end{namedthm*}

	\begin{namedthm*}{Sequential Criterion for Uniform Continuity}
		The function \(f: I \rightarrow \mathbb{R}\) is uniformly continuous on I if and only if for any two sequences \(\left(x_{n}\right)\) and \(\left(y_{n}\right)\)
		in \(I\) such that \(x_{n}-y_{n} \rightarrow 0,\) we have \(f\left(x_{n}\right)-f\left(y_{n}\right) \rightarrow 0\). Corollary: f not continuous on I if you find two sequences \(\left(x_{n}\right)\) and \(\left(y_{n}\right)\) in \(I\) such that \(x_{n}-y_{n} \rightarrow 0\) but \(f\left(x_{n}\right)-f\left(y_{n}\right) \rightarrow 0\).
	\end{namedthm*}

	\begin{namedthm*}{Theorem 5.5.3}
		If \(f\) is continuous on a closed bounded interval \([a, b],\) then it is uniformly continuous
		on \([a, b] .\)
	\end{namedthm*}

	\begin{namedthm*}{Theorem 5.5.4}
		If I is an interval and \(f: I \rightarrow \mathbb{R}\) satisfies the Lipschitz condition on I, that is, there
		is a \(K>0\) such that
		\[
			|f(x)-f(y)| \leq K|x-y|, \quad \forall x, y \in I
		\]
		then \(f\) is uniformly continuous on \(I .\)
	\end{namedthm*}

	\begin{namedthm*}{Theorem 5.5.5}
		If \(\left.f: I \rightarrow \text { R is uniformly continuous on I and ( } x_{n}\right)\) is a Cauchy sequence in I, then
		\(\left(f\left(x_{n}\right)\right)\) is a Cauchy sequence.
	\end{namedthm*}

	\begin{namedthm*}{Theorem 5.5.6}
		If the function \(f:(a, b) \rightarrow \mathbb{R}\) is uniformly continuous on \((a, b),\) then \(f(a)\) and \(f(b)\)
		can be defined so that the extended function is continuous on \([a, b] .\) Take a sequence \(\left(x_{n}\right)\) in \((a, b)\) such that \(x_{n} \rightarrow a\). Define \(f(a)=\lim _{n \rightarrow \infty} f\left(x_{n}\right)\). Same for b.
	\end{namedthm*}
	% \section{Past Midterm Questions}
	% \begin{namedthm*}{18/19 Question 5}
	% 	If \(\left(x_{n}\right)\) converges to 0 and \(\left(y_{n}\right)\) is bounded, then \(\left(x_{n} y_{n}\right)\) converges to \(0 .\)\\
	% 	\textbf{Solution:} Since \(\left(y_{n}\right)\) is bounded, there exists \(M>0\) such that \(\left|y_{n}\right| \leq M\) for all \(n \in \mathbb{N} .\) Now let \(\varepsilon>0 .\) since \(x_{n} \rightarrow 0\) , there exists \(K \in \mathbb{N}\) such that
	% 	\[
	% 		\left|x_{n}-0\right|<\frac{\varepsilon}{M} \quad \forall n \geq K
	% 	\]
	% 	Then
	% 	\[
	% 		n \geq K \Longrightarrow\left|x_{n} y_{n}-0\right|=\left|x_{n} \| y_{n}\right| \leq \frac{\varepsilon}{M} \cdot M=\varepsilon
	% 	\]
	% \end{namedthm*}

	% \begin{namedthm*}{18/19 Question 6}
	% 	Suppose \((a_n)\) is convergent and \(a = \lim_{n \rightarrow \infty}{a_n}\). For each \(n \in \mathbb{N}\), let
	% 	$$b_{n} = \frac{1}{n^2}\sum_{j=1}^{n}(n-j+1)a_{j} = \frac{na_{1}+(n-1)a_{2}+\cdots+2a_{n-1}+a_{n}}{n^2}$$.
	% 	\begin{enumerate}
	% 		\item Prove that for each \(n \in \mathbb{N}\),$$b_{n} = \frac{1}{n^2}\sum_{j=1}^{n}(n-j+1)(a_{j}-a) + \frac{n+1}{2n}a$$.\\Proof skipped.
	% 		\item Prove that \((b_{n})\) converges.\\
	% 		      \textbf{Solution.} Let \(c_{n} = \frac{1}{n^2}\sum_{j=1}^{n}(n-j+1)(a_{j}-a)\) such that \(b_{n} = c_{n} + \frac{n+1}{2n}a\). We want to prove that \(\lim_{n \rightarrow \infty}{c_n} = 0\).
	% 		      \\Let \(\epsilon > 0\). Since \(\lim_{n \rightarrow \infty}{a_n}=a\), \(\exists K_1 \in \mathbb{N}\) s.t.
	% 		      $$n \geq K_{1} \rightarrow |a_{n}-a| < \frac{\epsilon}{2}$$.
	% 		      \\Let \(C = \sum_{j=1}^{K_1}|a_j-a|\). By A.P \(\exists K_2 \in \mathbb{N}\) s.t. \(K_2 > \frac{2C}{\epsilon}\). Let \(K = \max(K_1,K_2)\). Then for \(n \geq K\), we have
	% 		      \begin{align*}
	% 			      |c_n - 0| & =  \left|\frac{1}{n^2}\sum_{j=1}^{K_1}(n-j+1)(a_{j}-a)\right|                                              \\
	% 			                & \leq  \frac{1}{n^2}\sum_{j=1}^{K_1}(n-j+1)|a_{j}-a|                                                        \\ & ~~~~~+ \frac{1}{n^2}\sum_{j=K_{1}+ 1}^{n}(n-j+1)|a_{j}-a|\\
	% 			                & \leq \frac{1}{n^2}\sum_{j=1}^{K_1}n|a_{j}-a| + \frac{1}{n^2}\sum_{j=K_{1}+ 1}^{n}n\cdot \frac{\epsilon}{2} \\
	% 			                & \leq \frac{C}{n} + \frac{1}{n^2}(n-K_{1})n\frac{\epsilon}{2}                                               \\
	% 			                & \leq \frac{C}{K} + \frac{\epsilon}{2}                                                                      \\
	% 			                & < \frac{C}{2C/\epsilon} + \frac{\epsilon}{2} = \epsilon
	% 		      \end{align*}
	% 	\end{enumerate}
	% \end{namedthm*}

	% \begin{namedthm*}{13/14 Question 6}
	% 	Let \((a_n)\) and \((b_n)\) be defined by setting $$a_1 = 3, b_1 = 2, a_{n+1} = a_n+ 2b_n \text{ and } b_{n+1}= a_n + b_n \text{ for } n \in \mathbb{N}$$. Moreover, let \(c_n = \frac{a_n}{b_n}\).
	% 	\begin{enumerate}
	% 		\item Express \(c_{n+1}\) in terms of \(c_{n}\).
	% 		      \\\textbf{Solution.} \begin{align*}
	% 			      c_{n+1} & =\frac{a_{n+1}}{b_{n+1}}
	% 			      \\&=\frac{a_n+2b_{n}}{a_{n}+b_{n}}
	% 			      \\&=\frac{\frac{a_n}{b_n} + 2}{\frac{a_n}{b_n} + 1}
	% 			      \\&=\frac{c_n + 2}{c_n+1}
	% 		      \end{align*}
	% 		\item Prove that \(\left|c_{n+1}-\sqrt{2}\right| < r\left|c_{n}-\sqrt{2}\right|\), \(r= \sqrt{2}-1\).
	% 		      \\\textbf{Solution.} \begin{align*}
	% 			      \left|c_{n+1}-\sqrt{2}\right| & =\left|\frac{c_n + 2}{c_n + 1}-\sqrt{2}\right|
	% 			      \\&=\frac{1}{c_n + 1}\left|\left(1-\sqrt{2}\right)c_{n} + \left(2- \sqrt{2}\right)\right|
	% 			      \\&=\frac{1}{c_n + 1}\left|\left(1-\sqrt{2}\right)\left(c_n- \sqrt{2}\right)\right|
	% 			      \\&=\frac{\sqrt{2}-1}{c_n + 1}\left|c_{n} - \sqrt{2}\right|
	% 			      \\&=\frac{r}{c_n + 1}\left|c_{n} - \sqrt{2}\right|
	% 			      \\& < r\left|c_{n} - \sqrt{2}\right|
	% 		      \end{align*}
	% 		\item Prove that \(c_{n}\) converges and find its limit.
	% 		      \\\textbf{Solution.} For \(n \geq 2\) we have by (ii),
	% 		      $$\left|c_n-\sqrt{2} \right|< r\left|c_{n-1}-\sqrt{2} \right|<\cdots<r^{n-1}\left|c_{1}-\sqrt{2} \right|$$
	% 		      Since \(0 < r < 1\), \(e^{n-1} \rightarrow 0\). So $$r^{n-1}\left|c_{1}-\sqrt{2} \right| \rightarrow 0.$$ By the Squeeze Theorem, \(\left|c_{n}-\sqrt{2} \right| \rightarrow 0\). It follows that \(\lim_{n \rightarrow \infty}{c_n} = \sqrt{2}\).
	% 	\end{enumerate}
	% \end{namedthm*}
	% \section{Tutorial Questions}
	% \begin{namedthm*}{Ratio theorem from Tutorial 4}
	% 	If $$\lim_{n \rightarrow \infty}{\frac{x_{n+1}}{x_n}}=L$$ where \(L < 1\), then $$\lim_{n \rightarrow \infty}{x_n} = 0$$.
	% \end{namedthm*}
	% \begin{namedthm*}{Expansion of \(x^n - y^n\)}
	% 	$$x^n - y^n = (x-y)(x^{p-1}+x^{p-2}y+x^{p-3}y^{2}+ \cdots +xy^{p-2}+y^{p-1})$$
	% \end{namedthm*}

	% \begin{namedthm*}{Tutorial 3 Question 2d}
	% 	Prove $$\lim_{n \rightarrow \infty}{\frac{4^n}{n!}} = 0$$.\\
	% 	\textbf{Solution.} Notice that
	% 	$$\frac{4^n}{n!} = \frac{4}{1}\cdot\frac{4}{2}\cdot\frac{4}{3}\cdot\underbrace{\frac{4}{4}\cdot\frac{4}{5}\cdots\cdot\frac{4}{n-1}}_\text{\(\leq 1\)}\cdot\frac{4}{n}$$ when \(n \geq 5\). So $$\frac{4^n}{n!} \leq \frac{256}{6n} < \frac{43}{n}$$. Let \(\epsilon > 0\). By A.P, \(\exists K \in \mathbb{N}\) s.t. \(K > \max(5,\frac{43}{\epsilon})\)...
	% \end{namedthm*}


	% \begin{namedthm*}{Tutorial 1 Question 3}
	% 	For any \(n \in \mathbb{N}, n > 1\),$$\left(1+\frac{1}{n-1}\right)^{n-1}<\left(1+\frac{1}{n}\right)^{n}$$
	% \end{namedthm*}

	% \begin{namedthm*}{Tutorial 4 Question 1g}

	% \end{namedthm*}

	% \begin{namedthm*}{Homework 2 Question 2}
	% 	Let \(S=\left\{\frac{m}{2 n+3 m} : n, m \in \mathbb{N}\right\} .\) Prove that \(\sup S=1 / 3\) and inf \(S=0\)
	% 	\\\textbf{Solution.} First show that \(\frac{1}{3}\) is an upper bound. Then, let \(\varepsilon>0 .\) By the Archimedean Property, there is a natural number \(m_{0}\) such that \(m_{0}>\frac{1}{\varepsilon} .\) Let
	% 	\(x_{\varepsilon}=\frac{m_{0}}{2+3 m_{0}} .\) Then
	% 	\[
	% 		x_{\varepsilon} \in S \quad \text { and } \quad x_{\varepsilon}=\frac{m_{0}}{2+3 m_{0}}=\frac{1}{3}-\frac{2}{3\left(2+3 m_{0}\right)}>\frac{1}{3}-\frac{1}{m_{0}}>\frac{1}{3}-\varepsilon
	% 	\]
	% 	By Lemma \(1.9 .1,\) sup \(S=\frac{1}{3} .\)\\
	% 	On the other hand, 0 is clearly a lower bound of \(S\) . Let \(\varepsilon>0 .\) By the Archimedean Property, there
	% 	is a natural number \(n_{0}\) such that \(n_{0}>\frac{1}{\varepsilon} .\) Let \(y_{\varepsilon}=\frac{1}{2 n_{0}+3} .\) Then
	% 	\[
	% 		y_{\varepsilon} \in S \quad \text { and } \quad y_{\varepsilon}=\frac{1}{2 n_{0}+3}<\frac{1}{n_{0}}<0+\varepsilon
	% 	\]
	% 	By the result of Question \(\mathrm{H} 2,\) inf \(S=0\)
	% \end{namedthm*}

	% \begin{namedthm*}{Results from Tutorial 2}~
	% 	\begin{enumerate}
	% 		\item If \(0<a<b,\) then \(a^{n}<b^{n}\) for every \(n \in \mathbb{N}\)
	% 		\item If \(0<a<b\) and \(r \in \mathbb{Q}\) with \(r>0,\) then \(a^{r}<b^{r}\)
	% 		\item If \(A\) and \(B\) are bounded nonempty subsets of \(\mathbb{R}\) then \(\sup (A \cup B)=\max (\sup A, \sup B)\)
	% 	\end{enumerate}
	% \end{namedthm*}

	% \begin{namedthm*}{Tutorial 3 Question 7}
	% 	If \(\lim _{n \rightarrow \infty} x_{n}=x\) , then
	% 	\[
	% 		\lim _{n \rightarrow \infty} \frac{x_{1}+x_{2}+\cdots+x_{n}}{n}=x
	% 	\]
	% \end{namedthm*}



\end{multicols}

\end{document}