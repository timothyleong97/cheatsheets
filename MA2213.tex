\documentclass[10pt,landscape]{article}
\usepackage{amssymb,amsmath,amsthm,amsfonts,bm}
\usepackage{multicol,multirow}
\usepackage{calc}
\usepackage{ifthen}
\usepackage[landscape]{geometry}
\usepackage[colorlinks=true,citecolor=blue,linkcolor=blue]{hyperref}
\ifthenelse{\lengthtest { \paperwidth = 11in}}
    { \geometry{top=.2in,left=.2in,right=.2in,bottom=.2in} }
	{\ifthenelse{ \lengthtest{ \paperwidth = 297mm}}
		{\geometry{top=1cm,left=1cm,right=1cm,bottom=1cm} }
		{\geometry{top=1cm,left=1cm,right=1cm,bottom=1cm} }
	}
\pagestyle{empty}
\makeatletter
\renewcommand{\section}{\@startsection{section}{1}{0mm}%
                                {-1ex plus -.5ex minus -.2ex}%
                                {0.5ex plus .2ex}%x
                                {\normalfont\large\bfseries}}
\renewcommand{\subsection}{\@startsection{subsection}{2}{0mm}%
                                {-1explus -.5ex minus -.2ex}%
                                {0.5ex plus .2ex}%
                                {\normalfont\normalsize\bfseries}}
\renewcommand{\subsubsection}{\@startsection{subsubsection}{3}{0mm}%
                                {-1ex plus -.5ex minus -.2ex}%
                                {1ex plus .2ex}%
                                {\normalfont\small\bfseries}}
\makeatother
\setcounter{secnumdepth}{0}
\setlength{\parindent}{0pt}
\setlength{\parskip}{0pt plus 0.5ex}



\theoremstyle{definition}
\newcommand{\thistheoremname}{}
\newtheorem*{genericthm*}{\thistheoremname}
\newenvironment{namedthm*}[1]
{\renewcommand{\thistheoremname}{#1}\begin{genericthm*}}
{\end{genericthm*}}

% -----------------------------------------------------------------------

\title{MA2213 Cheatsheet 19/20 Sem 1 Midterm}

\begin{document}

\begin{center}
	{\large MA2213 Cheatsheet 19/20 Sem 1 Midterm}\\{by Timothy Leong (format by Ning Yuan)}
\end{center}

\raggedright
\footnotesize

\begin{multicols}{3}

	\setlength{\premulticols}{1pt}
	\setlength{\postmulticols}{1pt}
	\setlength{\multicolsep}{1pt}
	\setlength{\columnsep}{2pt}

	\section{Lecture 8: Lagrange Interpolation}
	\begin{namedthm*}{Solving via linear system}
		\[\begin{bmatrix}
				1      & x_0    & x_0^2  & \cdots & x_0^n  \\
				1      & x_1    & x_1^2  & \cdots & x_1^n  \\
				\vdots & \vdots & \vdots & \ddots & \vdots \\
				1      & x_n    & x_n^2  & \cdots & x_n^n  \\
			\end{bmatrix}
			\begin{bmatrix}
				a_0    \\
				a_1    \\
				\vdots \\
				a_n
			\end{bmatrix}
			=
			\begin{bmatrix}
				f(x_0) \\
				f(x_1) \\
				\vdots \\
				f(x_n)
			\end{bmatrix}
		\]
	\end{namedthm*}

	\begin{namedthm*}{Solving via basis polynomials} Let
		\(L_{k}(x)=\frac{\left(x-x_{0}\right) \cdots\left(x-x_{k-1}\right)\left(x-x_{k+1}\right) \cdots\left(x-x_{n}\right)}{\left(x_{k}-x_{0}\right) \cdots\left(x_{k}-x_{k-1}\right)\left(x_{k}-x_{k+1}\right) \cdots\left(x_{k}-x_{n}\right)}\) for \(k=0,1, \cdots, n\). Then \(P_{n}(x)=y_{0} L_{0}(x)+y_{1} L_{1}(x)+\cdots+y_{n} L_{n}(x)\).
	\end{namedthm*}

	\begin{namedthm*}{Error analysis}
		\(x_{0}, x_{1}, \cdots, x_{n}\) are distinct in \([a, b]\)
		\(\implies \forall x \in[a, b], \exists \xi \in\left(\min \left\{x, x_{0}, x_{1}, \cdots, x_{n}\right\}, \max \left\{x, x_{0}, x_{1}, \cdots, x_{n}\right\}\right)\) such that
		\[
			f(x)=P_{n}(x)+\frac{f^{(n+1)}(\xi)}{(n+1) !}\left(x-x_{0}\right)\left(x-x_{1}\right) \cdots\left(x-x_{n}\right)
		\] 
	\end{namedthm*}
	\section{Tutorial 3: Lagrange Interpolation}
	\section{Lecture 9: Divided Differences}
	\begin{namedthm*}{Structure of Lagrange polynomial}
		\(
			P_{n}(x)=a_{0}+a_{1}\left(x-x_{0}\right)+a_{2}\left(x-x_{0}\right)\left(x-x_{1}\right)+\cdots+a_{n}\left(x-x_{0}\right)\left(x-x_{1}\right) \cdots\left(x-x_{n-1}\right)
        \)
        where \(a_{k}=f\left[x_{0}, x_{1}, \cdots, x_{k}\right]\) and \(f\left[x_{0}, x_{1}, \cdots, x_{n}\right]=\frac{f\left[x_{1}, x_{2}, \cdots, x_{n}\right]-f\left[x_{0}, x_{1}, \cdots, x_{n-1}\right]}{x_{n}-x_{0}}\). \(a_0 = f(x_0)\).
    \end{namedthm*}
	\section{Lecture 10: Cubic Spline Interpolation}
	\section{Tutorial 4: Divided diff and CSI}
	\section{Lecture 11: Least Squares Approximation}
	\section{Lecture 12: Newton-Cotes Formulae}
	\section{Lecture 13: Composite Numerical Integration}
	\section{Tutorial 5: LSA and Integration}
	\section{Miscellaneous}
	\begin{namedthm*}{The Gamma function} \(\Gamma(x)=\int_{0}^{+\infty} t^{x-1} \mathrm{e}^{-t} \mathrm{d} t\) and \(\Gamma(n)=(n-1) !\) for any positive integer \(n\).
	\end{namedthm*}



\end{multicols}
\end{document}