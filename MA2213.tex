% Credits to Ning Yuan for the format

\documentclass[10pt,landscape]{article}
\usepackage{amssymb,amsmath,amsthm,amsfonts,bm}
\newenvironment{amatrix}[1]{%
  \left(\begin{array}{@{}*{#1}{c}|c@{}}
}{%
  \end{array}\right)
}
\usepackage{multicol,multirow}
\usepackage{calc}
\usepackage{ifthen}
\usepackage[landscape]{geometry}
\usepackage[colorlinks=true,citecolor=blue,linkcolor=blue]{hyperref}
\ifthenelse{\lengthtest { \paperwidth = 11in}}
    { \geometry{top=.2in,left=.2in,right=.2in,bottom=.2in} }
	{\ifthenelse{ \lengthtest{ \paperwidth = 297mm}}
		{\geometry{top=1cm,left=1cm,right=1cm,bottom=1cm} }
		{\geometry{top=1cm,left=1cm,right=1cm,bottom=1cm} }
	}
\pagestyle{empty}
\makeatletter
\renewcommand{\section}{\@startsection{section}{1}{0mm}%
                                {-1ex plus -.5ex minus -.2ex}%
                                {0.5ex plus .2ex}%x
                                {\normalfont\large\bfseries}}
\renewcommand{\subsection}{\@startsection{subsection}{2}{0mm}%
                                {-1explus -.5ex minus -.2ex}%
                                {0.5ex plus .2ex}%
                                {\normalfont\normalsize\bfseries}}
\renewcommand{\subsubsection}{\@startsection{subsubsection}{3}{0mm}%
                                {-1ex plus -.5ex minus -.2ex}%
                                {1ex plus .2ex}%
                                {\normalfont\small\bfseries}}
\makeatother
\setcounter{secnumdepth}{0}
\setlength{\parindent}{0pt}
\setlength{\parskip}{0pt plus 0.5ex}



\theoremstyle{definition}
\newcommand{\thistheoremname}{}
\newtheorem*{genericthm*}{\thistheoremname}
\newenvironment{namedthm*}[1]
{\renewcommand{\thistheoremname}{#1}\begin{genericthm*}}
{\end{genericthm*}}

% -----------------------------------------------------------------------

\title{MA2213 AY19/20 Semester 1 Finals}

\begin{document}

\begin{center}
	{\large MA2213 AY19/20 Semester 1 Finals}
\end{center}

\raggedright
\footnotesize

\begin{multicols}{3}

	\setlength{\premulticols}{1pt}
	\setlength{\postmulticols}{1pt}
	\setlength{\multicolsep}{1pt}
	\setlength{\columnsep}{2pt}

	\section{Lecture 8: Lagrange Interpolation}
	\begin{namedthm*}{Solving via linear system} \label{ge}
		\[\begin{bmatrix}
				1      & x_0    & x_0^2  & \cdots & x_0^n  \\
				1      & x_1    & x_1^2  & \cdots & x_1^n  \\
				\vdots & \vdots & \vdots & \ddots & \vdots \\
				1      & x_n    & x_n^2  & \cdots & x_n^n  \\
			\end{bmatrix}
			\begin{bmatrix}
				a_0    \\
				a_1    \\
				\vdots \\
				a_n
			\end{bmatrix}
			=
			\begin{bmatrix}
				f(x_0) \\
				f(x_1) \\
				\vdots \\
				f(x_n)
			\end{bmatrix}
		\]
	\end{namedthm*}

	\begin{namedthm*}{Solving via basis polynomials} Let
		\(L_{k}(x)=\frac{\left(x-x_{0}\right) \cdots\left(x-x_{k-1}\right)\left(x-x_{k+1}\right) \cdots\left(x-x_{n}\right)}{\left(x_{k}-x_{0}\right) \cdots\left(x_{k}-x_{k-1}\right)\left(x_{k}-x_{k+1}\right) \cdots\left(x_{k}-x_{n}\right)}\) for \(k=0,1, \cdots, n\). Then \(P_{n}(x)=y_{0} L_{0}(x)+y_{1} L_{1}(x)+\cdots+y_{n} L_{n}(x)\).
	\end{namedthm*}

	\begin{namedthm*}{Error analysis}
		\(\forall x \in[a, b], \exists \xi \in\left(\min \left\{x, x_{0}, x_{1}, \cdots, x_{n}\right\}, \max \left\{x, x_{0}, x_{1}, \cdots, x_{n}\right\}\right)\) such that
		\[
			f(x)=P_{n}(x)+\frac{f^{(n+1)}(\xi)}{(n+1) !}\left(x-x_{0}\right)\left(x-x_{1}\right) \cdots\left(x-x_{n}\right)
		\]
	\end{namedthm*}
	\section{Tutorial 3: Lagrange Interpolation}
	\begin{namedthm*}{Example 1} Construct the LIP for \(f(x)=\log _{2} x\) using \(x_{0}=1 / 2, x_{1}=1, x_{2}=2, x_{3}=4 .\) Find a bound of the absolute error for any
		\(x \in[1 / 2,+\infty) .\)
		\\\textbf{Solution.} \(f\left(0.5\right)=-1, \quad f\left(1\right)=0, \quad f\left(2\right)=1, \quad f\left(4\right)=2\). Solve using linear system
		to get \(a_0= -\frac{52}{21}\), \(a_1 = \frac{7}{2}\), \(a_2 =-\frac{7}{6}\), \(a_3=\frac{1}{7}\). So \(P(x)=\frac{1}{7} x^{3}-\frac{7}{6} x^{2}+\frac{7}{2} x-\frac{52}{21}\). To get an u.b for error, first find \(f^{(4)}(x)=-\frac{6}{x^{4} \ln 2}\). Note monotonicity. Hence \(\left|f^{(4)}(x)\right| \leqslant \frac{6}{(1 / 2)^{4} \ln 2}=\frac{96}{\ln 2}\). Thus an u.b for absolute error \(|P(x)-f(x)|\) is
		\(
		\frac{1}{4 !} \times \frac{96}{\ln 2}|(x-1 / 2)(x-1)(x-2)(x-4)|=\frac{4}{\ln 2}|(x-1 / 2)(x-1)(x-2)(x-4)|. \medspace \square
		\).
	\end{namedthm*}

	\begin{namedthm*}{Result from Example 2}
		If nodes are equidistributed, the maximum value of \(g(x)=\left|\left(x-x_{0}\right)\left(x-x_{1}\right) \cdots\left(x-x_{N}\right)\right|\) must be attained in \(\left(x_{0}, x_{1}\right)\) and \(\left(x_{N-1}, x_{N}\right)\) (due to the symmetry). \(\left|g\left(x^{*}\right)\right| \leqslant \frac{1}{4} N ! h^{N+1}\).
	\end{namedthm*}
	\begin{namedthm*}{Error estimation for equidistributed nodes}
		\(\left|P_{N}(x)-f(x)\right| \leqslant \frac{h^{N+1}}{4(N+1)} \max_{\xi \in[a, b]}\left|f^{(N+1)}(\xi)\right|,\) for all \(x \in[a, b]\)
	\end{namedthm*}

	\begin{namedthm*}{Exercise 2} Let \(P_{n}(x)\) be the LIP for \(f(x)=\cos x\) with \(x_{k}=k h, \quad k=0,1,\cdots, n\) where \(h=\pi /(2 n)\). \textbf{1.} Find a positive integer \(N\) such that
		\(
		\left|P_{N}(x)-f(x)\right|<0.005, \quad \text { for all } x \in[0, \pi / 2].
		\)
		\textbf{Solution.} For \(f(x)=\cos(x)\), \(\max _{\xi \in[0, \pi / 2]}\left|f^{(N+1)}(\xi)\right|=1\). Hence it suffices to find \(\frac{h^{N+1}}{4(N+1)} < 0.005 \implies N \geq 3\).
	\end{namedthm*}
	\section{Lecture 9: Divided Differences}
	\begin{namedthm*}{How to find the Lagrange polynomial}
		\(
		P_{n}(x)=a_{0}+a_{1}\left(x-x_{0}\right)+a_{2}\left(x-x_{0}\right)\left(x-x_{1}\right)+\cdots+a_{n}\left(x-x_{0}\right)\left(x-x_{1}\right) \cdots\left(x-x_{n-1}\right)
		\)
		where \(a_{k}=f\left[x_{0}, x_{1}, \cdots, x_{k}\right]\) and \(f\left[x_{0}, x_{1}, \cdots, x_{n}\right]=\frac{f\left[x_{1}, x_{2}, \cdots, x_{n}\right]-f\left[x_{0}, x_{1}, \cdots, x_{n-1}\right]}{x_{n}-x_{0}}\). \(a_0 = f(x_0)\).
	\end{namedthm*}
	\section{Lecture 10: Cubic Spline Interpolation (CSI)}

	\begin{namedthm*}{How to find \(\mu_k \text{ and } \lambda_k\)}\(\mu_{k}=\frac{x_{k}-x_{k-1}}{x_{k+1}-x_{k-1}}, \quad \lambda_{k}=\frac{x_{k+1}-x_{k}}{x_{k+1}-x_{k-1}}, \quad k=1,2, \cdots, n-1\)
	\end{namedthm*}

	\begin{namedthm*}{Natural Boundary Conditions}\(M_0 = M_n = 0\).
		\[
			\begin{bmatrix}
				2     & \lambda_1 &           &           &               \\
				\mu_2 & 2         & \lambda_2 &           &               \\
				      & \mu_3     & 2         & \ddots    &               \\
				      &           & \ddots    & \ddots    & \lambda_{n-2} \\
				      &           &           & \mu_{n-1} & 2
			\end{bmatrix}
			\begin{bmatrix}
				M_1     \\
				M_2     \\
				M_3     \\
				\vdots  \\
				M_{n-2} \\
				M_{n-1} \\
			\end{bmatrix} =
			\begin{bmatrix}
				6f[x_0,x_1,x_2]             \\
				6f[x_1,x_2,x_3]             \\
				6f[x_2,x_3,x_4]             \\
				\vdots                      \\
				6f[x_{n-3},x_{n-2},x_{n-1}] \\
				6f[x_{n-2},x_{n-1}, x_n]    \\
			\end{bmatrix}
		\]
	\end{namedthm*}

	\begin{namedthm*}{Clamped Boundary Conditions}\(2 M_{0}+M_{1}=6 f\left[x_{0}, x_{0}, x_{1}\right], \quad M_{n-1}+2 M_{n}=6 f\left[x_{n-1}, x_{n}, x_{n}\right]\).
		\[
			\begin{bmatrix}
				2     & \lambda_0 &           &        &               \\
				\mu_1 & 2         & \lambda_1 &        &               \\
				      & \ddots    & \ddots    & \ddots &               \\
				      &           & \mu_{n-1} & 2      & \lambda_{n-1} \\
				      &           &           & \mu_n  & 2
			\end{bmatrix}
			\begin{bmatrix}
				M_0     \\
				M_1     \\
				\vdots  \\
				M_{n-1} \\
				M_{n}
			\end{bmatrix} =
			\begin{bmatrix}
				6f[x_0,x_0,x_1]           \\
				6f[x_0,x_1,x_2]           \\
				\vdots                    \\
				6f[x_{n-2},x_{n-1},x_{n}] \\
				6f[x_{n-1},x_{n}, x_n]
			\end{bmatrix}
		\]
	\end{namedthm*}

	\begin{namedthm*}{How to find \(S_k\)} \(S_{k}(x)=M_{k-1} \frac{\left(x-x_{k}\right)^{3}}{6\left(x_{k-1}-x_{k}\right)}+M_{k} \frac{\left(x-x_{k-1}\right)^{3}}{6\left(x_{k}-x_{k-1}\right)}+A_{k} x+B_{k}\). \(A_{k}=\frac{f\left(x_{k}\right)-f\left(x_{k-1}\right)}{x_{k}-x_{k-1}}-\frac{1}{6}\left(M_{k}-M_{k-1}\right)\left(x_{k}-x_{k-1}\right)\).\\ \(B_{k}=\frac{x_{k} f\left(x_{k-1}\right)-x_{k-1} f\left(x_{k}\right)}{x_{k}-x_{k-1}}+\frac{1}{6}\left(M_{k} x_{k-1}-M_{k-1} x_{k}\right)\left(x_{k}-x_{k-1}\right)\).
	\end{namedthm*}
	\begin{namedthm*}{Piecewise Linear Interpolation (PLI)} If \(S(x) = S_k(x), \quad \text {for } x\in [x_{k-1}, x_k], k = 1,2,\cdots, n\) then
		\(S_k(x)=f(x_{k-1}) \frac{x-x_k}{x_k-x_{k-1}}+f(x_k) \frac{x-x_{k-1}}{x_k-x_{k-1}}\)
	\end{namedthm*}
	\begin{namedthm*}{Error analysis for PLI on equidistributed nodes}
		If \(x_{k}=x_{0}+k h\), then for \(x \in [x_0,x_n]\), \(|f(x)-S(x)| \leq \frac{1}{8}h^2\max_{\xi \in [x_0,x_n]}|f''(\xi)|\)
	\end{namedthm*}


	\section{Tutorial 4: Divided Diff and CSI}
	\begin{namedthm*}{Example 2: Quadratic spline interpolation}
		Given \(n+1\) nodes \(x_{0}<x_{1}<\cdots<x_{n-1}<x_{n}\) and a continuous function \(f(x),\)
		find a function \(S(x)\) such that
		\textbf{1.} \(S(x)\) is first-order differentiable on \(\left(x_{0}, x_{n}\right)\)
		\textbf{2.} \(S(x)\) is a quadratic polynomial on \(\left(x_{k-1}, x_{k}\right)\) for any \(k=2,3, \cdots, n ;\)
		\textbf{3.} \(\left.S(x) \text { is a linear function on (} x_{0}, x_{1}\right)\). \textbf{4.} \(S\left(x_{k}\right)=f\left(x_{k}\right)\) for all \(k=0,1, \cdots, n\).
		\\\textbf{Solution.} \(S_{1}(x)=f\left(x_{0}\right) \frac{x-x_{1}}{x_{0}-x_{1}}+f\left(x_{1}\right) \frac{x-x_{0}}{x_{1}-x_{0}}\). \(S_{k}(x)=\frac{1}{2} M_{k} \frac{\left(x-x_{k-1}\right)^{2}}{x_{k}-x_{k-1}}+\frac{1}{2} M_{k-1} \frac{\left(x-x_{k}\right)^{2}}{x_{k-1}-x_{k}}+C_{k}, \quad k=2,3, \cdots, n.\)\\\(C_{k}=f\left(x_{k}\right)-\frac{1}{2} M_{k}\left(x_{k}-x_{k-1}\right)\). \(M_{k}=2 f\left[x_{k-1}, x_{k}\right]-M_{k-1}\). Since \(M_{1}=S_{1}^{\prime}\left(x_{1}\right)=f\left[x_{0}, x_{1}\right]\), \(M_{k}\) for \(k=2,3, \cdots, n\) can be obtained iteratively.
	\end{namedthm*}
	\section{Lecture 11: Least Squares Approximation}
	\begin{namedthm*}{Proof of optimality}
		Suppose a satisfies \(
		X^{T} X \mathbf{a}=X^{T} \mathbf{y}.\) Then for any vector b with the same length as a, we have
		\(
		(X \mathbf{b}-\mathbf{y})^{T}(X \mathbf{b}-\mathbf{y}) \geqslant(X \mathbf{a}-\mathbf{y})^{T}(X \mathbf{a}-\mathbf{y})\). \textbf{Proof.}\[
		\begin{aligned}&(X \mathbf{b}-\mathbf{y})^{T}(X \mathbf{b}-\mathbf{y})-(X \mathbf{a}-\mathbf{y})^{T}(X \mathbf{a}-\mathbf{y}) \\=& \mathbf{b}^{T} X^{T} X \mathbf{b}-2 \mathbf{b}^{T} X^{T} \mathbf{y}-\mathbf{a}^{T} X^{T} X \mathbf{a}+2 \mathbf{a}^{T} X^{T} \mathbf{y} \\=& \mathbf{b}^{T} X^{T} X \mathbf{b}-2 \mathbf{b}^{T} X^{T} X \mathbf{a}-\mathbf{a}^{T} X^{T} X \mathbf{a}+2 \mathbf{a}^{T} X^{T} X \mathbf{a} \\=& \mathbf{b}^{T} X^{T} X \mathbf{b}-2 \mathbf{b}^{T} X^{T} X \mathbf{a}+\mathbf{a}^{T} X^{T} X \mathbf{a}=(\mathbf{b}-\mathbf{a})^{T} X^{T} X(\mathbf{b}-\mathbf{a}) \geqslant 0 \end{aligned}
		\]

	\end{namedthm*}
	\begin{namedthm*}{Finding the coefficients}
		Let \(X=\left(\begin{array}{ccccc}{1} & {x_{0}} & {x_{0}^{2}} & {\cdots} & {x_{0}^{n}} \\ {1} & {x_{1}} & {x_{1}^{2}} & {\cdots} & {x_{1}^{n}} \\ {\vdots} & {\vdots} & {\vdots} & {} & {\vdots} \\ {1} & {x_{m}} & {x_{m}^{2}} & {\cdots} & {x_{m}^{n}}\end{array}\right)\) and
		\\ \(\mathbf{a}=\left(a_{0}, a_{1}, \cdots, a_{n}\right)^{T}, \quad \mathbf{y}=\left(y_{0}, y_{1}, \cdots, y_{m}\right)^{T}\). Solve \(X^TXa = X^Ty.\)
	\end{namedthm*}

	\begin{namedthm*}{Lecture Exercise 1} Show that the least squares approximation is unique if and only if the matrix \(X\)
		has full column rank, \(i.e.,\) the rank of \(X\) equals its number of columns. \textbf{Proof.} Note \(\operatorname{rank}(X^TX)=\operatorname{rank}(X)\). \((\implies)\)If the LSA is unique, the columns are linearly independent since we can write \(Xa\) as a unique linear combination of the columns of \(X\). Hence the rank of the matrix is equal to the number of columns.\((\impliedby)\) If \(Xa = b \text{ and } Xa' = b\) then \(X(a-a')=0\). Since X has full rank, \(Xc = 0\) iff \(c = 0 \implies a = a'\) (only the trivial solution to the homogeneous equation of linear combination of its columns exists).
	\end{namedthm*}
	\begin{namedthm*}{Weighted LSA}
		\(W=\operatorname{diag}\left\{w_{0}, w_{1}, \cdots, w_{n}\right\}\). Solve \(X^{T} W X \mathbf{a}=X^{T} W \mathbf{y}\).
	\end{namedthm*}
	\section{Lecture 12: Newton-Cotes Formulae}
	\begin{namedthm*}{Ascertaining number of points for error \(< \epsilon\)}
		\color{red} TODO
	\end{namedthm*}
	\begin{namedthm*}{Trapezoidal Rule}\(\int_{a}^{b} f(x) \mathrm{d} x \approx \frac{b-a}{2}[f(a)+f(b)]\).
	\end{namedthm*}
	\begin{namedthm*}{Error for Trapezoidal Rule}\(\int_{a}^{b} f(x) \mathrm{d} x=\frac{b-a}{2}[f(a)+f(b)]-\frac{1}{12}(b-a)^{3} f^{\prime \prime}(\xi)\)
	\end{namedthm*}
	\begin{namedthm*}{Simpson's Rule} Given \(f(a), f\left(\frac{a+b}{2}\right)\) and \(f(b)\), \(P(x)=f(a)+\frac{f(b)-f(a)}{b-a}(x-a)+\left[2 f\left(\frac{a+b}{2}\right)-f(b)-f(a)\right] \frac{2(x-a)(x-b)}{(b-a)^{2}}\) whose integral is \(\left[\frac{2}{3} f\left(\frac{a+b}{2}\right)+\frac{1}{6} f(b)+\frac{1}{6} f(a)\right](b-a)\).
	\end{namedthm*}
	\begin{namedthm*}{Error for Simpson's Rule}\(\int_{a}^{b} f(x) \mathrm{d} x=\frac{b-a}{6}\left[f(a)+4 f\left(\frac{a+b}{2}\right)+f(b)\right]-\frac{1}{90}\left(\frac{b-a}{2}\right)^{5} f^{(4)}(\xi)\)
	\end{namedthm*}
	\begin{namedthm*}{Theorem 3 on Page 4}
		\color{red}TODO
	\end{namedthm*}
	\begin{namedthm*}{Gaussian Elimination to find weights}
		\color{red}TODO
	\end{namedthm*}
	\section{Lecture 13: Composite Numerical Integration}
	\begin{namedthm*}{Composite Trapezoidal Rule (CTR)}
		\color{red} Confusing use of h. Read carefully first
	\end{namedthm*}
	\begin{namedthm*}{Error analysis for CTR}
		\color{red} TODO
	\end{namedthm*}
	\begin{namedthm*}{Composite Simpson's Rule (CSR)}
		\color{red} TODO
	\end{namedthm*}
	\begin{namedthm*}{Error analysis for CSR}
		\color{red} TODO
	\end{namedthm*}
	\section{Tutorial 5: LSA and Integration}
	\begin{namedthm*}{Figuring out the data points used in LSA} If the \(p(x)\) is given but the data points are incomplete, form the \(X^TXa=X^Ty\) system and select rows that look very similar to eliminate as many variables at once as possible.
	\end{namedthm*}

	\section{Miscellaneous}
	\begin{namedthm*}{The Gamma function} \(\Gamma(x)=\int_{0}^{+\infty} t^{x-1} \mathrm{e}^{-t} \mathrm{d} t\) and \(\Gamma(n)=(n-1) !\) for any positive integer \(n\).
	\end{namedthm*}
	\begin{namedthm*}{Linear Algebra and Calculus}
		\color{red} TODO
	\end{namedthm*}
	\begin{namedthm*}{General formulae for higher derivatives}
		\hyperref{https://www.math24.net/higher-order-derivatives/}{}{}{Higher order derivatives}
	\end{namedthm*}
	\begin{namedthm*}{Error proofs from lectures/tutorials}
		\color{red} LSA proof of optimality, the proof of errors, tutorial identities
	\end{namedthm*}
	\begin{namedthm*}{Finding the set of x and y values for the lowest error LIP}
		First find the lowest-error polynomial (LEP), which is \(p(x)=a_{0}+a_{1} x+a_{2} x^{2}+\cdots+a_{m} x^{m}\). From Homework Q3, \(M=\left(\begin{array}{cccc}{b-a} & {\frac{b^{2}-a^{2}}{2}} & {\cdots} & {\frac{b^{m+1}-a^{m+1}}{m+1}} \\ {\frac{b^{2}-a^{2}}{2}} & {\frac{b^{3}-a^{3}}{3}} & {\cdots} & {\frac{b^{m+2}-a^{m+2}}{m+2}} \\ {\vdots} & {\vdots} & {} & {\vdots} \\ {\frac{b^{m+1}-a^{m+1}}{m+1}} & {\frac{b^{m+2}-a^{m+2}}{m+2}} & {\cdots} & {\frac{b^{2 m+1}-a^{2 m+1}}{2 m+1}}\end{array}\right)\) and b = \(\left(\int_{a}^{b} x^{0} f(x) d x, \cdots, \int_{a}^{b} x^{m} f(x) d x\right)^{T}\). Solve \(Ma = b\) to get the coefficients of the LEP. Then reverse-engineer the Gaussian Elimination process for finding \(a\) when constructing the LIP to get x and y (possibly non-unique).
	\end{namedthm*}

\end{multicols}
\end{document}