\documentclass[10pt,landscape]{article}
\usepackage{amssymb,amsmath,amsthm,amsfonts,bm}
\usepackage{multicol,multirow}
\usepackage{calc}
\usepackage{ifthen}
\usepackage[landscape]{geometry}
\usepackage[colorlinks=true,citecolor=blue,linkcolor=blue]{hyperref}
\usepackage{graphicx}
\graphicspath{ {./images/} }

\ifthenelse{\lengthtest { \paperwidth = 11in}}
    { \geometry{top=.2in,left=.2in,right=.2in,bottom=.2in} }
	{\ifthenelse{ \lengthtest{ \paperwidth = 297mm}}
		{\geometry{top=1cm,left=1cm,right=1cm,bottom=1cm} }
		{\geometry{top=1cm,left=1cm,right=1cm,bottom=1cm} }
	}
\pagestyle{empty}
\makeatletter
\renewcommand{\section}{\@startsection{section}{1}{0mm}%
                                {-1ex plus -.5ex minus -.2ex}%
                                {0.5ex plus .2ex}%x
                                {\normalfont\large\bfseries}}
\renewcommand{\subsection}{\@startsection{subsection}{2}{0mm}%
                                {-1explus -.5ex minus -.2ex}%
                                {0.5ex plus .2ex}%
                                {\normalfont\normalsize\bfseries}}
\renewcommand{\subsubsection}{\@startsection{subsubsection}{3}{0mm}%
                                {-1ex plus -.5ex minus -.2ex}%
                                {1ex plus .2ex}%
                                {\normalfont\small\bfseries}}
\makeatother
\setcounter{secnumdepth}{0}
\setlength{\parindent}{0pt}
\setlength{\parskip}{0pt plus 0.5ex}



\theoremstyle{definition}
\newcommand{\thistheoremname}{}
\newtheorem*{genericthm*}{\thistheoremname}
\newenvironment{namedthm*}[1]
{\renewcommand{\thistheoremname}{#1}\begin{genericthm*}}
{\end{genericthm*}}

% -----------------------------------------------------------------------

\title{MA2213 Cheatsheet 19/20 Sem 1 Midterm}

\begin{document}

\begin{center}
{\large MA2213 Cheatsheet 19/20 Sem 1 Midterm}\\{by Timothy Leong (format taken from Ning Yuan)}
\end{center}

\raggedright
\footnotesize

\begin{multicols}{3}

\setlength{\premulticols}{1pt}
\setlength{\postmulticols}{1pt}
\setlength{\multicolsep}{1pt}
\setlength{\columnsep}{2pt}

\section{Lecture 2: Computer Arithmetic}

\begin{namedthm*}{Single-precision floating point format}
$$\underbrace{b_{31}}_\text{sign}\underbrace{b_{30}\cdots b_{23}}_\text{8-bit exp}\underbrace{b_{22}\cdots b_{0}}_\text{23-bit mantissa}$$ represents the binary number $$(-1)^{b_{31}} \times 10^{b_{30} b_{29} \cdots b_{23} b_{23}-11111111} \times 1 . b_{22} \cdots b_{1}$$
\end{namedthm*}

\begin{namedthm*}{Example: Converting from decimal to IEEE}
$$(9.15625)_{10}=(1001.00101)_{2}=\left(1.00100101 \times 10^{11}\right)_{2}$$
The mantissa is 00100101000000000000000 and the exponent is
$$
(11+1111111)_{2}=(10000010)_{2}
$$
Note: The exponent is in the excess-127 format
\end{namedthm*}



\begin{namedthm*}{Exceptions to the IEEE format}
\includegraphics[scale=0.25]{exceptions.png}
\\In fact, when the exponent is 11111111 and the mantissa is not zero, the result is always
interpreted as NaN or -NaN, which means "Not-a-Number".
\end{namedthm*}


\begin{namedthm*}{Denormal Numbers}
If exponent is all zeros, the binary expression for denormal numbers is 
$$(-1)^{b_{31}} \times 10^{-1111110} \times 0 . b_{22} b_{21} \cdots b_{1} b_{0}$$
and the corresponding decimal number is $$(-1)^{b_{31}} \times 2^{-126} \times \sum_{i=1}^{23} b_{23-i} 2^{-i}$$
\end{namedthm*}


\section{Lecture 4: Linear Systems \& Cramer's Rule}


\begin{namedthm*}{Matrix-vector multiplication}
\makeatletter\includegraphics[scale=0.5]{mat-vec-mult.png}
\textbf{Time complexity: $O(n^2)$}
\end{namedthm*}

\begin{namedthm*}{Cramer's rule}
Suppose $A$ is an $n \times n$ matrix with non-zero determinant. Then
the unique solution of $A \mathbf{x}=\mathbf{b}$ is given by
$$
x_{i}=\frac{\operatorname{det} A_{i}}{\operatorname{det} A}, \quad i=1, \cdots, n
$$
where $A_{t}$ is the matrix performed by replacing the $i$th column of $A$ by the column vector
b.
\textbf{Time complexity: $O(n^3)$}
\end{namedthm*}

\section{Lecture 5: Gaussian Elimination}
\begin{namedthm*}{Gaussian Elimination with no pivoting}
The first 8 lines are elimination steps, and lines 9-16 are for backward substitution.
\includegraphics[scale=0.5, width=8cm]{gaussian-no-pivot.png}
\textbf{Time complexity: $O(n^3)$}
\end{namedthm*}


\begin{namedthm*}{Gaussian Elimination with partial pivoting (naive)}
\includegraphics[scale=0.5, width=8cm]{gaussian-with-naive-pivoting.png} This algorithm only performs a swap when $a_{r_{k}j}=0$. However, imprecise subtraction can cause a zero pivot to be non-zero.
\end{namedthm*}


\section{Lecture 6: Pivoting strategy}

\begin{namedthm*}{Gaussian Elimination with partial pivoting (more reliable)}
 We replace lines 5-15 in the above algorithm with this block of code, setting the entry with the largest magnitude as the pivot. \textbf{Time complexity: O(n^3).}\\
$$
     \includegraphics[width=5cm]{gaussian-partial-pivot-improved.png}
$$
\end{namedthm*}

\begin{namedthm*}{Gaussian Elimination with scaled partial pivoting (even more reliable)}
First, we locate the largest-magnitude number in each row and set that number as the scaling factor for that row.
\includegraphics[scale=0.7]{images/scaling_factors.png}
\includegraphics[scale=0.5, width=8.5cm]{images/row_swap_partial_pivoting.png}
\end{namedthm*}

\section{Lecture 7: Matrix Factorization}

\begin{namedthm*}{Types of row operations}
~
\begin{enumerate}
    \item $E_{i} \leftarrow E_{i}-m_{i j} E_{j}$ (multiply the $j$th row by $m_{i j}$ and subtract the result from the $i$th
$\text { row }),$ where $i>j$. This is equivalent to multiplying
the augmented matrix by the following matrix on the left-hand side: $$
I-m_{i j} \mathrm{e}_{i} \mathrm{e}_{j}^{T}
$$ where $I$ is the identity matrix, and $\mathrm{e}_{i}$ is the unit vector $(0, \cdots, 0,1,0, \cdots, 0)^{T},$ whose the only
nonzero entry is its $i$th component. When $i>j$, the elements of the matrix $I-m_{i j} \mathbf{e}_{i} \mathbf{e}_{j}^{T}$ can
be described as follows:
\begin{itemize}
    \item The matrix is lower-triangular;
    \item All its diagonal entries are equal to 1;
    \item All its off-diagonal entries are zero except that its $(i, j)$ -element is $-m_{i j}$.
\end{itemize}
\item $\left.E_{i} \leftrightarrow E_{j} \text { (exchange the } i \text {th row and the } j \text {th row}\right)$ which is equivalent to pre-multiplying
$$
I-\left(\mathbf{e}_{i}-\mathbf{e}_{j}\right)\left(\mathbf{e}_{i}-\mathbf{e}_{j}\right)^{T}
$$
Features:\begin{itemize}
    \item The matrix is symmetric;
    \item All its diagonal entries are equal to 1 except the $i$th and $j$th ones are equal to zero;
    \item All its off-diagonal entries are zero except that its $(i, j)$-element and $(j, i)$-element are equal to 1.
\end{itemize}

\end{enumerate}
\end{namedthm*}

\begin{namedthm*}{Theorem 1}
% if no space, take out all the words, leave the picture
Suppose $j \in\{1, \cdots, n-1\}$ and $i_{1}, i_{2}, \cdots, i_{k}$ are integers in $\{j+1, \cdots, n\} .$ Then
$$
\begin{aligned}(I-&\left.m_{i 1} j \mathbf{e}_{i_{1}} \mathbf{e}_{j}^{T}\right)\left(I-m_{i_{2} j} \mathbf{e}_{i_{2}} \mathbf{e}_{j}^{T}\right) \cdots\left(I-m_{i_{k} j} \mathbf{e}_{i_{k}} \mathbf{e}_{j}^{T}\right) \\ &=I-\left(m_{i_{1} j} \mathbf{e}_{i_{1}}+m_{i_{2} j} \mathbf{e}_{i_{2}}+\cdots+m_{i_{k} j} \mathbf{e}_{i_{k}}\right) \mathbf{e}_{j}^{T} \end{aligned}
$$. When $i_{1}, i_{2}, \cdots, i_{k}$ are distinct, the matrix
$$
I-\left(m_{i_{1} j} \mathbf{e}_{i_{1}}+m_{i_{2} \mathbf{j}} \mathbf{e}_{i_{2}}+\cdots+m_{i_{k} \mathbf{j}} \mathbf{e}_{i_{k}}\right) \mathbf{e}_{j}^{T}
$$ has the following structure:
\begin{itemize}
    \item All the diagonal entries are equal to 1
    \item The $\left(i_{s}, j\right)$ -element is $m_{i_{i} s}$ for any $s=1, \cdots, k$
    \item All other entries are equal to zero.
\end{itemize}
Or pictorially,
$$
\left(\begin{array}{cccc}{1} & {0} & {0} & {0} \\ {0} & {1} & {0} & {0} \\ {0} & {0} & {1} & {0} \\ {1 / 3} & {0} & {0} & {1}\end{array}\right)\left(\begin{array}{cccc}{1} & {0} & {0} & {0} \\ {0} & {1} & {0} & {0} \\ {-1 / 3} & {0} & {1} & {0} \\ {0} & {0} & {0} & {1}\end{array}\right)\left(\begin{array}{cccc}{1} & {0} & {0} & {0} \\ {-2 / 3} & {1} & {0} & {0} \\ {0} & {0} & {1} & {0} \\ {0} & {0} & {0} & {1}\end{array}\right)\\=\left(\begin{array}{rrrr}{1} & {0} & {0} & {0} \\ {-2 / 3} & {1} & {0} & {0} \\ {-1 / 3} & {0} & {1} & {0} \\ {1 / 3} & {0} & {0} & {1}\end{array}\right)
$$\\
If $A \mathrm{x}=\mathrm{b}$ can be solved by GE without pivoting, then the GE is equivalent to pre-multiplying the matrix $A$ by $n-1$ lower triangular matrices, and each lower triangular matrix has the form $I-\mathbf{m}_{j} \mathbf{e}_{j}^{T}$, and the process is equivalent to $\left(I-\mathbf{m}_{n-1} \mathbf{e}_{n-1}^{T}\right) \cdots\left(I-\mathbf{m}_{2} \mathbf{e}_{2}^{T}\right)\left(I-\mathbf{m}_{1} \mathbf{e}_{1}^{T}\right) A=U$, where $U$ is an upper triangular matrix.
\end{namedthm*}


\begin{namedthm*}{Theorem 2}
Suppose $j<n .$ Let $\mathbf{m}_{j}$ be the $n$ -dimensional column vector $\left(0, \cdots, 0, m_{j+1, j}, \cdots, m_{n j}\right)^{T}$. Then $\left(I-\mathbf{m}_{j} \mathbf{e}_{j}^{T}\right)^{-1}=I+\mathbf{m}_{j} \mathbf{e}_{j}^{T}$, i.e.
$\left(\begin{array}{rrrr}{1} & {0} & {0} & {0} \\ {-2 / 3} & {1} & {0} & {0} \\ {-1 / 3} & {0} & {1} & {0} \\ {1 / 3} & {0} & {0} & {1}\end{array}\right)^{-1}=\left(\begin{array}{rrrr}{1} & {0} & {0} & {0} \\ {2 / 3} & {1} & {0} & {0} \\ {1 / 3} & {0} & {1} & {0} \\ {-1 / 3} & {0} & {0} & {1}\end{array}\right)$
\end{namedthm*}



\begin{namedthm*}{Theorem 3}
Suppose $i_{1} \leqslant i_{2} \leqslant \cdots \leqslant i_{k}<n,$ and for any $j=1, \cdots, k,$ the vector $\mathbf{m}_{j}$ is an $n$-dimensional column vector whose first $i_{j}$ components are zero. Then
$\left(I+\mathbf{m}_{1} \mathbf{e}_{i_{1}}^{T}\right)\left(I+\mathbf{m}_{2} \mathbf{e}_{i_{2}}^{T}\right) \cdots\left(I+\mathbf{m}_{k} \mathbf{e}_{i_{k}}^{T}\right)=I+\mathbf{m}_{1} \mathbf{e}_{i_{1}}^{T}+\mathbf{m}_{2} \mathbf{e}_{i_{2}}^{T}+\cdots \mathbf{m}_{k} \mathbf{e}_{i_{k}}^{T}$
\end{namedthm*}
\begin{namedthm*}{Example of Theorem 3}
$\left(\begin{array}{rrrrr}{1} & {0} & {0} & {0} \\ {2 / 3} & {1} & {0} & {0} \\ {1 / 3} & {0} & {1} & {0} \\ {-1 / 3} & {0} & {0} & {1}\end{array}\right)\left(\begin{array}{rrrr}{1} & {0} & {0} & {0} \\ {0} & {1} & {0} & {0} \\ {0} & {4 / 5} & {1} & {0} \\ {0} & {1} & {0} & {1}\end{array}\right)\left(\begin{array}{llll}{1} & {0} & {0} & {0} \\ {0} & {1} & {0} & {0} \\ {0} & {0} & {1} & {0} \\ {0} & {0} & {5} & {1}\end{array}\right)$ $=$  $\left(\begin{array}{rrrr}{1} & {0} & {0} & {0} \\ {2 / 3} & {1} & {0} & {0} \\ {1 / 3} & {4 / 5} & {1} & {0} \\ {-1 / 3} & {1} & {5} & {1}\end{array}\right)$.
\end{namedthm*}


\begin{namedthm*}{Theorem 4}
If $Ax=b$ can be solved by GE without pivoting, $A$ can be factorised into L and U. $Ax=b \leftrightarrow LUx=b$. We can solve $Ly=b$ by forward substitution and then $Ux=y$ by backward substitution.
\includegraphics[width = 8cm]{images/Ly=b.png}
\includegraphics[width = 8cm]{images/Ux=y.png}
\\The time complexity of solving a linear system by LU-factorization is $O(n^2)$
\end{namedthm*}

\section{Tutorials}
\begin{namedthm*}{Tut 1 Exercise 3}
$$B(x, y)=\sum_{n=0}^{+\infty}(-1)^{n} \frac{(y-1) \cdots(y-n)}{n !(x+n)}$$
Given $x$ and $y,$ write down the pseudocode to approximate the value of $B(x, y)$\\
\includegraphics[height=3cm]{images/tut1-ex3.png}
\end{namedthm*}


\begin{namedthm*}{Tut 2 Exercise 2}
Let $A=\left(a_{i j}\right)_{n \times n}$ be an upper-triangular matrix and $B=\left(b_{i j}\right)_{n \times n}$ be a lower-
triangular matrix. Write an algorithm to compute $C=A B$ and count the number of arithmetic
operations.
\includegraphics[width=8.5cm]{images/tut2-ex2-ans.png}
\end{namedthm*}
\begin{namedthm*}{Tut 2 Exercise 3}
Write the pseudocode that uses GE without pivoting to
solve the linear system $A \mathrm{x}=\mathrm{b}$ with $\mathrm{b}=\left(b_{1}, \cdots, b_{n}\right)^{T}$ and $A$ being an upper-Hessenberg matrix
\includegraphics[scale=0.2]{images/upper_hessenberg.png}
\includegraphics[width=8cm]{images/tut2-ex3-ans.png} The idea is to only do a row subtraction with the row immediately below.
\end{namedthm*}




% just extra knowledge
\section{Miscellaneous}
\begin{namedthm*}{Summation formulae}
$$\includegraphics[scale=0.7]{sum_formulae.png}$$
\\Also from tutorial 2
$$\sum_{i=1}^{n}\left(\left(n + \frac{1}{2}i\right)-\frac{1}{2}i^2\right)=\frac{n(n+1)(2n+1)}{6}$$
\end{namedthm*}

% \begin{namedthm*}{Determinants}~
% \begin{enumerate}
%     \item $\mathbf{det}(A^T)=\mathbf{det}(A)$
%     \item $\mathbf{det}(cA)=c^{n}\mathbf{det}(A)$
%     \item $\mathbf{det}(AB)=\mathbf{det}(A)\mathbf{det}(B)$
%     \item $\mathbf{det}(A^{-1})=1/\mathbf{det}(A)$
% \end{enumerate}
% \end{namedthm*}



\end{multicols}
\end{document}